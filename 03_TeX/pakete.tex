
\usepackage[ngerman]{babel}

% Kodierung fuer unixoide Systeme und Windowssysteme.

\usepackage[utf8]{inputenc}

% Schriftart Times New Roman.

%\usepackage{lmodern}
%\usepackage{times}
%\usepackage{newtxtext,newtxmath}
\usepackage{mathptmx}

% Erweitert den Zeichenvorrat, so dass z.B. auch Umlaute im PDF-Dokument
% gefunden werden.

\usepackage[T1]{fontenc}

%Bibliography Fehlerbehebung
\usepackage{bibgerm}

% Zur Einbindung von Bildern.

\usepackage{graphicx}

% Erweiterte enumerate-Umgebung.

\usepackage{enumerate}


% Verschiedene Pakete, die nuetzlich sind um Mathematik in LaTeX zu setzen.

\usepackage{mathtools}
\mathtoolsset{showonlyrefs=true}
\usepackage{amsmath, amssymb, amsthm, dsfont}
\usepackage{mathrsfs}
\usepackage{listings}
\usepackage{xcolor}
\definecolor{dunkelgruen}{HTML}{0B6121}
\lstset{
language=R,                     % the language of the code
  basicstyle=\tiny, % the size of the fonts that are used for the code
  numbers=left,                   % where to put the line-numbers
  numberstyle=\tiny\color{gray},  % the style that is used for the line-numbers
  stepnumber=1,                   % the step between two line-numbers. If it's 1, each line
                                  % will be numbered
  numbersep=5pt,                  % how far the line-numbers are from the code
 % backgroundcolor=\color{white},  % choose the background color. You must add \usepackage{color}
  showspaces=false,               % show spaces adding particular underscores
  showstringspaces=false,         % underline spaces within strings
  showtabs=false,                 % show tabs within strings adding particular underscores
%  frame=single,                   % adds a frame around the code
%  rulecolor=\color{black},        % if not set, the frame-color may be changed on line-breaks within not-black text (e.g. commens (green here))
  tabsize=2,                      % sets default tabsize to 2 spaces
  captionpos=t,                   % sets the caption-position to bottom
    breaklines=true,                % sets automatic line breaking
  breakatwhitespace=false,        % sets if automatic breaks should only happen at whitespace
%  title=\lstname,                 % show the filename of files included with \lstinputlisting;
  otherkeywords={},                                % also try caption instead of title
  deletekeywords={off,Q,t,q,file,data},
  keywordstyle=\color{blue},      % keyword style
  commentstyle=\color{dunkelgruen},   % comment style
 stringstyle=\color{violet},      % string literal style
 %  escapeinside={\%*}{*)},         % if you want to add a comment within your code
%  morekeywords={*,...},           % if you want to add more keywords to the set
%alsoletter={!},  
  literate={ö}{{\"o}}1
           {ä}{{\"a}}1
           {ü}{{\"u}}1
           {ß}{{\ss}}1
}

% Doppelte 1 (\mathbbm{1})

\usepackage{bbm}

% Indentation

%\usepackage{scrextend}

% Spaces for newcommand

%\usepackage{xspace}

% tikz
\usepackage{tikz}
%\usetikzlibrary{arrows.meta}% arrow tip library
%\usetikzlibrary{bending}% better arrow head for bended lines

% Hyphenation of words that already contain a hyphen
% http://tex.stackexchange.com/questions/2706/adequate-hyphenation-of-words-already-containing-a-hyphen
% Example:
% \textsc{Alexandroff}\hyp{}\textsc{Urysohn}\hyp{}Kompaktheit

%\usepackage{hyphenat}


% Hyperlinks

%\usepackage{hyperref}
\usepackage[pdfauthor={Adrian Gabel},%
pdftitle={Bachelorarbeit - Bewertung Amerikanischer Optionen durch Lösen eines optimalen Stoppproblems},%
%pagebackref=true,%
%pdftex
]{hyperref}
%\hypersetup{ colorlinks   =  false}
\usepackage[a-1b]{pdfx}
%\usepackage{url}
%\def\UrlBreaks{\do\/\do-}

%\usepackage[
%%nonumberlist, %keine Seitenzahlen anzeigen
%toc,          %Einträge im Inhaltsverzeichnis
%section]      %im Inhaltsverzeichnis auf section-Ebene erscheinen
%{glossaries}

%%%%%%%%%%%%%%%%
% Seitenlayout %
%%%%%%%%%%%%%%%%

% DIV# gibt den Divisor für die Layoutberechnung an.
% Vergrößern des Divisors vergrößert den Textbereich.
% BCOR#cm gibt die Breite des Bundstegs an.
\usepackage[DIV14,BCOR2cm]{typearea}

% Abstand obere Blattkante zur Kopfzeile ist 2.54cm - 15mm
\setlength{\topmargin}{-15mm}

% Keine Einrueckung nach einem Absatz.

\parindent 0pt

% Abstand zwischen zwei Abs\"atzen.

%\parskip 12pt

% Zeilenabstand.

\linespread{1.25}

% Inhaltsverzeichnis erstellen.

\usepackage{pgfplots}
\pgfplotsset{compat=1.12}
\pgfplotsset{samples=50}

\usepgfplotslibrary{statistics}


\usepackage{multicol}

\usepackage{subcaption}
\usepackage{microtype}
\usepackage{parcolumns}
\usepackage{changepage}

\usepackage{fancyhdr}
\usepackage{booktabs}
\usepackage{caption}

\usepackage[utf8]{inputenc}
\usepackage{pgfplots}
\pgfplotsset{compat=newest}
\usepgfplotslibrary{groupplots}
\usepgfplotslibrary{dateplot}
\usepgfplotslibrary{patchplots}

\usepackage{filecontents}

\usetikzlibrary{matrix,chains,positioning,decorations.pathreplacing,arrows}
\usetikzlibrary{positioning,calc}
\tikzset{%
  every neuron/.style={circle,draw,minimum size=1cm},
  neuron missing/.style={draw=none,scale=4,text height=0.333cm,execute at begin node=\color{black}$\vdots$},
}


