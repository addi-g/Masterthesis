% Vorlage fuer Abschlussarbeiten am Fachbereich Mathematik der TU Darmstadt.
% Geeignet fuer Bachelorarbeiten, Masterarbeiten und Diplomarbeiten.

%\documentclass[11pt,a4paper,twoside,openright]{report}
\documentclass[12pt,a4paper,twoside]{article}
% Benutzt man die Option draft, so kann man die Umbrueche ueberpruefen.

% Hier werden alle benoetigten Pakete und Einstellungen geladen. Auch
% hier sind Sie frei diese direkt in der Praeambel zu laden.


\usepackage[ngerman]{babel}

% Kodierung fuer unixoide Systeme und Windowssysteme.

\usepackage[utf8]{inputenc}

% Schriftart Times New Roman.

%\usepackage{lmodern}
%\usepackage{times}
%\usepackage{newtxtext,newtxmath}
\usepackage{mathptmx}

% Erweitert den Zeichenvorrat, so dass z.B. auch Umlaute im PDF-Dokument
% gefunden werden.

\usepackage[T1]{fontenc}

%Bibliography Fehlerbehebung
\usepackage{bibgerm}

% Zur Einbindung von Bildern.

\usepackage{graphicx}

% Erweiterte enumerate-Umgebung.

\usepackage{enumerate}


% Verschiedene Pakete, die nuetzlich sind um Mathematik in LaTeX zu setzen.

\usepackage{mathtools}
%\mathtoolsset{showonlyrefs=false}
\usepackage{amsmath, amssymb, amsthm, dsfont}
\usepackage{mathrsfs}
\usepackage{listings}
\usepackage{xcolor}
\definecolor{dunkelgruen}{HTML}{0B6121}
\lstset{
language=R,                     % the language of the code
  basicstyle=\tiny, % the size of the fonts that are used for the code
  numbers=left,                   % where to put the line-numbers
  numberstyle=\tiny\color{gray},  % the style that is used for the line-numbers
  stepnumber=1,                   % the step between two line-numbers. If it's 1, each line
                                  % will be numbered
  numbersep=5pt,                  % how far the line-numbers are from the code
 % backgroundcolor=\color{white},  % choose the background color. You must add \usepackage{color}
  showspaces=false,               % show spaces adding particular underscores
  showstringspaces=false,         % underline spaces within strings
  showtabs=false,                 % show tabs within strings adding particular underscores
%  frame=single,                   % adds a frame around the code
%  rulecolor=\color{black},        % if not set, the frame-color may be changed on line-breaks within not-black text (e.g. commens (green here))
  tabsize=2,                      % sets default tabsize to 2 spaces
  captionpos=t,                   % sets the caption-position to bottom
    breaklines=true,                % sets automatic line breaking
  breakatwhitespace=false,        % sets if automatic breaks should only happen at whitespace
%  title=\lstname,                 % show the filename of files included with \lstinputlisting;
  otherkeywords={},                                % also try caption instead of title
  deletekeywords={off,Q,t,q,file,data},
  keywordstyle=\color{blue},      % keyword style
  commentstyle=\color{dunkelgruen},   % comment style
 stringstyle=\color{violet},      % string literal style
 %  escapeinside={\%*}{*)},         % if you want to add a comment within your code
%  morekeywords={*,...},           % if you want to add more keywords to the set
%alsoletter={!},  
  literate={ö}{{\"o}}1
           {ä}{{\"a}}1
           {ü}{{\"u}}1
           {ß}{{\ss}}1
}

% Doppelte 1 (\mathbbm{1})

\usepackage{bbm}

% Indentation

%\usepackage{scrextend}

% Spaces for newcommand

%\usepackage{xspace}

% tikz
\usepackage{tikz}
%\usetikzlibrary{arrows.meta}% arrow tip library
%\usetikzlibrary{bending}% better arrow head for bended lines

% Hyphenation of words that already contain a hyphen
% http://tex.stackexchange.com/questions/2706/adequate-hyphenation-of-words-already-containing-a-hyphen
% Example:
% \textsc{Alexandroff}\hyp{}\textsc{Urysohn}\hyp{}Kompaktheit

%\usepackage{hyphenat}


% Hyperlinks

%\usepackage{hyperref}
\usepackage[pdfauthor={Adrian Gabel},%
pdftitle={Bachelorarbeit - Bewertung Amerikanischer Optionen durch Lösen eines optimalen Stoppproblems},%
%pagebackref=true,%
%pdftex
]{hyperref}
%\hypersetup{ colorlinks   =  false}
\usepackage[a-1b]{pdfx}
%\usepackage{url}
%\def\UrlBreaks{\do\/\do-}

%\usepackage[
%%nonumberlist, %keine Seitenzahlen anzeigen
%toc,          %Einträge im Inhaltsverzeichnis
%section]      %im Inhaltsverzeichnis auf section-Ebene erscheinen
%{glossaries}

%%%%%%%%%%%%%%%%
% Seitenlayout %
%%%%%%%%%%%%%%%%

% DIV# gibt den Divisor für die Layoutberechnung an.
% Vergrößern des Divisors vergrößert den Textbereich.
% BCOR#cm gibt die Breite des Bundstegs an.
\usepackage[DIV14,BCOR2cm]{typearea}

% Abstand obere Blattkante zur Kopfzeile ist 2.54cm - 15mm
\setlength{\topmargin}{-15mm}

% Keine Einrueckung nach einem Absatz.

\parindent 0pt

% Abstand zwischen zwei Abs\"atzen.

%\parskip 12pt

% Zeilenabstand.

\linespread{1.25}

% Inhaltsverzeichnis erstellen.

\usepackage{pgfplots}
\pgfplotsset{compat=1.12}
\pgfplotsset{samples=50}

\usepgfplotslibrary{statistics}


\usepackage{multicol}

\usepackage{subcaption}
\usepackage{microtype}
\usepackage{parcolumns}
\usepackage{changepage}

\usepackage{fancyhdr}
\usepackage{booktabs}
\usepackage{caption}

\usepackage[utf8]{inputenc}
\usepackage{pgfplots}
\pgfplotsset{compat=newest}
\usepgfplotslibrary{groupplots}
\usepgfplotslibrary{dateplot}
\usepgfplotslibrary{patchplots}

\usepackage{filecontents}

\usetikzlibrary{matrix,chains,positioning,decorations.pathreplacing,arrows}
\usetikzlibrary{positioning,calc}
\tikzset{%
  every neuron/.style={circle,draw,minimum size=1cm},
  neuron missing/.style={draw=none,scale=4,text height=0.333cm,execute at begin node=\color{black}$\vdots$},
}



% Hier werden Makros und Umgebungen eingebunden. Diese werden separat
% in der Datei befehle.tex definiert. Sie sind frei diese Befehle auch
% direkt in der Praeambel zu definieren.
% Makros

\newcommand{\F}{\mathscr{F}}
\newcommand{\E}{\mathbb{E}\,}
\newcommand{\e}{\mathrm{e}\,}
\newcommand{\T}{\mathscr{T}}


\newcommand{\N}{\mathbb{N}} % natuerliche Zahlen
\newcommand{\Z}{\mathbb{Z}} % ganze Zahlen
\newcommand{\Q}{\mathbb{Q}} % rationale Zahlen
\newcommand{\R}{\mathbb{R}} % reelle Zahlen
\newcommand{\K}{\mathbb{K}} % Körper
\newcommand{\co}{\mathrm{co}} % compact open
\newcommand{\W}{\mathcal{W}} % Weyl-Gruppe
\newcommand{\1}{\mathbbm{1}} % Körper
\newcommand{\id}{\mathrm{id}}
\newcommand{\sq}{\mathrm{sq}}
\newcommand{\mult}{\mathrm{mult}}
\newcommand{\ReLU}{\mathrm{ReLU}}
\newcommand{\net}{\mathrm{net}}


%\newcommand{\C}{\mathcal{C}} % stetige Funktionen

\DeclareMathOperator{\Ret}{Re} % Realteil
%\newcommand{\PIT}{\textbf{PIT }} % Prime Ideal Theorem
\newcommand{\AC}{\textbf{AC}\xspace} % Prime Ideal Theorem
\newcommand{\UFT}{\textbf{UFT}\xspace} % Prime Ideal Theorem
\newcommand{\PIT}{\textbf{PIT}\xspace}
\newcommand{\ZF}{\textbf{ZF}\xspace}
\newcommand{\ZFC}{\textbf{ZFC}\xspace}
\newcommand{\CC}{\textbf{CC}\xspace}
\newcommand{\CCR}{\textbf{CC}($\R$)\xspace}
\newcommand{\PCC}{\textbf{PCC}\xspace}
\newcommand{\PCCR}{\textbf{PCC}($\R$)\xspace}

\newcommand{\grad}[1]{\gv{\nabla} #1} % for gradient
\let\divsymb=\div % rename builtin command \div to \divsymb
\renewcommand{\div}[1]{\mathrm{div\,} #1} % for divergence
\renewcommand{\d}[1]{\ensuremath\, {\operatorname{d}\!{#1}}}
\newcommand{\norm}[1]{\lVert #1 \rVert}

\newcommand{\alaoglu}{\textsc{Alaoglu}}
\newcommand{\banach}{\textsc{Banach}}
\newcommand{\cauchy}{\textsc{Cauchy}}
\newcommand{\euklid}{\textsc{Euklid}}
\newcommand{\fubini}{\textsc{Fubini}}
\newcommand{\hahn}{\textsc{Hahn}}
\newcommand{\hausdorff}{\textsc{Hausdorff}}
\newcommand{\helmholtz}{\textsc{Helmholtz}}
\newcommand{\hilbert}{\textsc{Hilbert}}
\newcommand{\hoelder}{\textsc{Hölder}}
\newcommand{\killing}{\textsc{Killing}}
\newcommand{\kondrachov}{\textsc{Kondrachov}}
\newcommand{\laplace}{\textsc{Laplace}}
\newcommand{\lebesgue}{\textsc{Lebesgue}}
\newcommand{\leibniz}{\textsc{Leibniz}}
\newcommand{\lie}{\textsc{Lie}}
\newcommand{\lipschitz}{\textsc{Lipschitz}}
\newcommand{\navier}{\textsc{Navier}}
\newcommand{\neumann}{\textsc{Neumann}}
\newcommand{\newton}{\textsc{Newton}}
\newcommand{\poincare}{\textsc{Poincaré}}
\newcommand{\rellich}{\textsc{Rellich}}
\newcommand{\riesz}{\textsc{Riesz}}
\newcommand{\schwartz}{\textsc{Schwartz}}
\newcommand{\sobolev}{\textsc{Sobolev}}
\newcommand{\stokes}{\textsc{Stokes}}
\newcommand{\stein}{\textsc{Stein}}
\newcommand{\urysohn}{\textsc{Urysohn}}

% Umgebungen für Definitionen, Sätze, usw.

%\newtheorem{defn}{Definition}[section]
%\newtheorem{ex}{Beispiel}[chapter]

\theoremstyle{plain}
\newtheorem{thm}{Satz}[chapter]%[section]
\newtheorem{lem}[thm]{Lemma}
\newtheorem{cor}[thm]{Korollar}
\newtheorem{prop}[thm]{Proposition}

\theoremstyle{definition}
\newtheorem{defn}[thm]{Definition}

\theoremstyle{remark}
\newtheorem*{bem}{Bemerkung}
\newtheorem*{beh}{Behauptung}
\newtheorem{bemnumber}[thm]{Bemerkung}

\def\Satzrefname{Satz}

\DeclareMathOperator{\spn}{span}
\DeclareMathOperator{\sign}{sign}
\DeclareMathOperator{\GL}{GL}
\DeclareMathOperator{\Sym}{Sym}
\DeclareMathOperator{\hgt}{ht}
\DeclareMathOperator{\dist}{dist}
\DeclareMathOperator{\loc}{loc}
\DeclareMathOperator{\supp}{supp}
\DeclareMathOperator{\diam}{diam}

%\renewcommand{\refname}{Literaturverzeichnis}
\addto\captionsngerman{%
\renewcommand*{\bibname}{Literaturverzeichnis}
}

\fancypagestyle{intro}{%
\fancyhf{} % clear all header and footer fields
\fancyfoot[LE,RO]{\thepage} % except the center
\renewcommand{\headrulewidth}{0.0pt}
\renewcommand{\footrulewidth}{0pt}
}

\fancypagestyle{plain}{%
\fancyhf{} % clear all header and footer fields
\fancyhead[LE,RO]{\thepage} % except the center
\renewcommand{\headrulewidth}{1.0pt}
\renewcommand{\footrulewidth}{0pt}
}

\fancypagestyle{MyStyle}{
\fancyhead{} % clear all header fields
\fancyhead[RE]{\normalfont \nouppercase{\leftmark}}
\fancyhead[LO]{\normalfont \nouppercase{\rightmark}}
\fancyfoot{} % clear all footer fields
\fancyhead[LE,RO]{\thepage}
\renewcommand{\headrulewidth}{1.0pt}
\renewcommand{\footrulewidth}{0.0pt}
%\renewcommand{\sectionmark}[1]{\markboth{{\normalfont{##1}}}{}}
\renewcommand{\sectionmark}[1]{\markboth{\thesection\quad {\normalfont{##1}}}{}}
}



\begin{document}

% Auf der Titelseite und im Inhaltsverzeichnis sollen keine
% Seitenzahlen dargestellt werden.
\pagestyle{empty}
\pagenumbering{arabic}
\setcounter{page}{-2}
% Hier wird die Titelseite eingebunden.
\begin{titlepage}
  \begin{center}
    \vspace{1cm}
    \includegraphics[width=0.5\linewidth]{TU_Darmstadt_Logo.pdf}
    \vspace{1cm}
    
    \large{Fachbereich Mathematik}
    \vspace{2.5cm}
    
    \large{Masterarbeit}
    \vspace{2cm}

    \Large{\textbf{Analyse der Konvergenzgeschwindigkeit eines \\ einfach berechenbaren Neuronale-Netze Schätzers}}
    
    \vspace*{3cm}    
    
		\large
                %\href{mailto:gabel@mathematik.tu-darmstadt.de}{Fabian Gabel}
                Adrian Gabel
    \vspace*{1.0cm}

    XX.0X.2020 \\
    \vspace*{2cm}

    Betreuer: Prof. Dr. Michael Kohler

    \vspace*{.5cm}

    %Zweiter Gutachter: Name des zweiten Gutachters\\[2ex]
    \vspace*{\fill}
    %\tiny{Version vom \today}
  \end{center}
\end{titlepage}
\vspace*{\fill}

\pagestyle{intro}
\pagenumbering{roman}

% Inhaltsverzeichnis erstellen.
\tableofcontents
% Ab sofort werden Seitenzahlen in der Kopfzeile angezeigt.
%\pagestyle{headings}

\cleardoublepage
\pagestyle{plain}
\pagenumbering{arabic}
%\chapter{Einleitung}
%\chapter*{Einleitung}\index{Einleitung}
\addcontentsline{toc}{chapter}{Einleitung}\index{Einleitung}

Erfinder träumen schon lange davon, Maschinen zu schaffen, die denken. Dieser Wunsch geht zumindest auf die Zeit des antiken Griechenlands zurück. Die mythischen Figuren Pygmalion, Daedalus und Hephaestus können alle als legendäre Erfinder interpretiert werden, und Galatea, Talos und Pandora können alle als künstliches Leben betrachtet werden (Ovid und Martin, 2004; Sparkes, 1996; Tandy, 1997).
Als programmierbare Computer zum ersten Mal konzipiert wurden, fragten sich die Menschen, ob solche Maschinen intelligent werden könnten, mehr als hundert Jahre bevor man sie baute (Lovelace, 1842). Heute ist die künstliche Intelligenz (KI) ein blühendes Feld mit vielen praktischen Anwendungen und aktiven Forschungsthemen. 
Künstliche Intelligenz ist längst in unserem Alltag präsent und dringt in immer mehr Bereiche vor. Sprachassistenten etwa sind bereits als Helfer auf dem Smartphone, im Auto oder zu Hause Normalität geworden. Fortschritte im Bereich der KI beruhen vor allem auf der Verwendung Neuronaler Netze. Vergleichbar mit der Funktionsweise des menschlichen Gehirns verknüpfen sie mathematisch definierte Einheiten miteinander.

Es besteht eine große Lücke zwischen den Schätzungen, für die schönen Konvergenzergebnisse die in der Theorie nachgewiesen wurden, und den Schätzungen, die in der Praxis verwendet werden können.

Maschinelle Lernverfahren können als Lernen einer Funktion $(f)$ zusammengefasst werden, die Eingangsvariablen $(X)$ auf Ausgangsvariablen $(Y$) abbildet.
$$Y = f(x)$$
Ein Algorithmus lernt diese Zielabbildungsfunktion aus Trainingsdaten.
Die Form der Funktion ist unbekannt, so dass es unsere Aufgabe als Praktiker des maschinellen Lernens ist, verschiedene Algorithmen des maschinellen Lernens zu evaluieren und zu sehen, welcher die zugrunde liegende Funktion besser annähert.
Unterschiedliche Algorithmen machen unterschiedliche Annahmen oder Vorurteile über die Form der Funktion und die Art und Weise, wie sie gelernt werden kann.

Algorithmen, die keine starken Annahmen über die Form der Abbildungsfunktion treffen, werden als nichtparametrische Algorithmen des maschinellen Lernens bezeichnet. Indem sie keine Annahmen treffen, können sie jede beliebige Funktionsform aus den Trainingsdaten lernen.
Nichtparametrische Methoden sind gut, wenn Sie über viele Daten und keine Vorkenntnisse verfügen und wenn Sie sich nicht allzu sehr um die Auswahl der richtigen Funktionen kümmern wollen. (Referenz - Artificial Intelligence: A Modern Approach, Seite 757)

Ziel dieser Arbeit ist es, die folgende Frage genauer zu betrachten: Wenn wir eine Regressionsschätzung des neuronalen Netzes theoretisch genau so definieren, wie sie in der Praxis umgesetzt wird, welches Konvergenzergebnis können wir dann für diese Schätzung vorweisen? 

Als erstes werden wir in Kapitel \ref{chap:1} grundlegende Definition und Hilfsresultate für den weiteren Verlauf der Arbeit sammeln.
Anschließend definieren wir in Kapitel \ref{chap:2} eine neue Regressionsschätzung für neuronale Netze, bei der die meisten Gewichte unabhängig von den Daten gewählt werden, die durch einige neuere Approximationsergebnisse für neuronale Netze motiviert sind, und die daher leicht zu implementieren ist. In Kapitel \ref{chap:3} zeigen wir dann unser Hauptresultat, dass wir für diese Schätzung Konvergenzraten ableiten können, falls die Regressionsfunktion glatt ist. 

Mit diesem Hauptergebnis ist es nun möglich einfach zu implementierende Regressionsverfahren für neuronale Netze zu definieren, die die gleiche Konvergenzrate wie lineare Regressionsschätzungen (z.B. Kernel- oder Spline-Schätzungen) erreichen, d.h. sie erreichen (bis zu einem logarithmischen Faktor) die optimale Minimax-Konvergenzrate $n-2p/(2p+d)$ im Falle einer ($p,C$)-glatten Regressionsfunktion, für jedes $p > 0$.
Abschließend wird die Leistung unseres neu vorgeschlagenen Schätzers für simulierte Daten in Kapitel \ref{chap:4} veranschaulicht. Diese Arbeit orientiert sich an \cite{kohler19}.

\newpage
\pagestyle{MyStyle}
%\chapter{Grundlagen}
%\input{optimalestoppzeit}
%\chapter{Regressionsbasierte Monte-Carlo-Verfahren}
%\input{rb_mc}

%\chapter*{Appendix}
\markboth{APPENDIX}{}
\addcontentsline{toc}{chapter}{Appendix}
\label{chap:app}

Der Programmcode ist wie folgt aufgebaut:
%\linespread{1.0}
\begin{itemize}
    \item[-] \texttt{main.py} ist das Hauptprogramm welches alle Schätzer aufruft und die Ouputs generiert.
    \item[-] \texttt{data\_gen.py} generiert die Daten die wir für unsere Simulation benötigen.
    \item[-] \texttt{help\_neural\_networks.py} fasst alle Hilfsfunktion zusammen.
    \item[-] \texttt{new\_neural\_network.py} enthält unseren Neuronale-Netze-Regressionsschätzer.
    \item[-] \texttt{fc\_neural\_network.py} enthält das fully connected neuronale Netz mit einer verborgenen Schicht.
    \item[-] \texttt{nearest\_neighbor.py} enthält einen Nächste-Nachbar Schätzer.
    \item[-] \texttt{constant.py} enthält den konstanten Schätzer.
\end{itemize}
\renewcommand\lstlistingname{Listing}
\lstinputlisting[basicstyle=\scriptsize, caption={\texttt{main.py}},captionpos=t]{../01_Code/main.py}
\lstinputlisting[basicstyle=\scriptsize, caption={\texttt{data\_gen.py}},captionpos=t]{../01_Code/data_gen.py}
\lstinputlisting[basicstyle=\scriptsize, caption={\texttt{help\_neural\_networks.py}},captionpos=t]{../01_Code/help_neural_networks.py}
\lstinputlisting[basicstyle=\scriptsize, caption={\texttt{new\_neural\_network.py}},captionpos=t]{../01_Code/new_neural_network.py}
\lstinputlisting[basicstyle=\scriptsize, caption={\texttt{fc\_neural\_network.py}},captionpos=t]{../01_Code/fc_neural_network.py}
\lstinputlisting[basicstyle=\scriptsize, caption={\texttt{nearest\_neighbor.py}},captionpos=t]{../01_Code/nearest_neighbor.py}
\lstinputlisting[basicstyle=\scriptsize, caption={\texttt{constant.py}},captionpos=t]{../01_Code/constant.py}
\newpage
\pagestyle{plain}
\nocite{*}
\bibliographystyle{geralpha}
\bibliography{Masterthesis_Adrian_Gabel}
\addcontentsline{toc}{section}{Literaturverzeichnis}


% Keine Kopf-/Fußzeile auf dieser Seite
\thispagestyle{empty}

\vspace*{4cm}
\section*{Erklärung}\index{Erklärung}

Hiermit versichere ich, dass ich die vorliegende Arbeit selbstständig verfasst habe und alle benutzten Quellen einschließlich der Quellen aus dem Internet und alle sonstigen Hilfsmittel angegeben habe.\vspace{20pt}

\noindent
Darmstadt, den XX.03.2020\vspace{60pt}

% Unterschrift (handgeschrieben)

\noindent
Adrian Gabel



\end{document}
