\chapter{Grundlagen und Hilfsresultate}
\label{chap:1}

Der Zweck dieses Kapitels ist es, grundlegende Definitionen zu sammeln, die in den folgenden Kapiteln verwendet werden. Weiterhin werden wir Hilfsresultate darstellen und beweisen welche wir vor allem für das Resultat der Konvergenzgeschwindigkeit des einfach berechenbaren Neuronale Netze Regresssionsschätzer benötigt werden.

\section{Definitionen}

\begin{defn}[Stoppzeit, vgl. \cite{Klenke2013} 9.15 und 9.16]\label{def:stop}
    \begin{itemize}
        \item[]
        \item[(i)]Eine \emph{Stoppzeit} $\tau$ mit Werten in $[0,T]$ und $T \geq 0$ ist eine bezüglich $(X_t)_{0\leq t \leq T}$ messbare Funktion. Zusätzlich muss für jedes $r \in [0,T]$ das Ereignis $\{\tau \leq r\}$ in der durch $(X_s)_{0 \leq s \leq r}$ erzeugten  $\sigma$-Algebra $\F_r = \sigma((X_s)_{0 \leq s \leq r})$ enthalten sein. Die Klasse aller Stoppzeiten mit Werten im Intervall $[0,T]$ bezeichnen wir mit $\T([0,T])$.
        \item[(ii)] Ist $T \in \N_0$, so nennen wir eine bezüglich $X_0,\dots,X_T$ messbare Funktion $\tau$ mit Werten in $\{0,1,\dots,T\}$, \emph{Stoppzeit}, wenn zusätzlich für jedes $r \in \{0,\dots,T\}$ das Ereignis $\{\tau = r\}$ in der durch $X_0,\dots,X_r$ erzeugten  $\sigma$-Algebra $\sigma(X_0,\dots,X_r)$ enthalten ist. Die Klasse aller Stoppzeiten mit Werten in $\{0,\dots,T\}$ bezeichnen wir mit $\T(0,\dots,T)$.
    \end{itemize}
\end{defn}

\section{Hilfsresultate}

\begin{lem}
  \label{lem:1}
  Sei $\sigma \colon \R \to \R$ eine Funktion und $R$,$a > 0$.
  \begin{itemize}
  \item[a)] Angenommen $\sigma$ ist zwei mal stetigdifferenzierbar und $t_{\sigma,id} \in \R$ so, dass $sigma'(t_{\sigma, id}) \neq 0$ ist. Dann gilt mit
  $$ f_{id}(x) = \frac{R}{\sigma'(t_{\sigma, id})} \cdot \left(\sigma\left(\frac{x}{R} + t_{\sigma, id}\right) - \sigma(t_{\sigma, id})\right)$$
  für beliebige $x \in [-a, a]\colon$ 
  $$ |f_{id}(x) - x| \leq \frac{\|\sigma''\|_{\infty} \cdot a^2}{2 \cdot |\sigma'(t_{\sigma, id})|} \cdot \frac{1}{R}.$$
  \item[b)] Angenommen $\sigma$ ist drei mal stetigdifferenzierbar und $t_{\sigma,sq} \in \R$ so, dass $sigma''(t_{\sigma, sq}) \neq 0$ ist. Dann gilt mit
  $$ f_{sq}(x) = \frac{R^2}{\sigma''(t_{\sigma, sq})} \cdot \left(\sigma\left(\frac{2 \cdot x}{R} + t_{\sigma, sq}\right) - 2 \cdot \sigma(\frac{x}{R} + t_{\sigma, sq})+ \sigma(t_{\sigma, sq})\right)$$
  für beliebige $x \in [-a, a]\colon$ 
  $$ |f_{sq}(x) - x^2| \leq \frac{5 \cdot \|\sigma'''\|_{\infty} \cdot a^3}{3 \cdot |\sigma'(t_{\sigma, sq})|} \cdot \frac{1}{R}.$$
  \end{itemize}
\end{lem}
\begin{proof}
	\begin{itemize}
  	\item[a)] Sei $u = \frac{c}{R} + t_{\sigma, id}$, $\xi = 0$ und  $x \in [-a, a]$beliebig. Wir wissen, dass $f_{id}$ 1-mal differenzierbar ist, da nach Vorraussetzung $\sigma$ 2-mal stetigdifferenzierbar ist,  existiert nach der Restgliedformel von Lagrange (REFERENZ) ein $c \in [\xi, x] $, sodass mit Ausklammern von $\frac{R}{\sigma'(t_{\sigma, id})} $ folgt$\colon
$
  	\begin{equation}
  	\begin{split}
  	 |f_{id}(x) -  & x| \\
  	& \leq \bigg|\frac{R}{\sigma'(t_{\sigma, id})} \cdot \bigg(\sigma\left(\frac{\xi}{R} + t_{\sigma, id}\right) - \sigma(t_{\sigma, id}) + \frac{1}{R} \sigma'\left(\frac{\xi}{R} + t_{\sigma, id}\right) (x - \xi) \\ & \qquad + \frac{1}{2R^2} \sigma''(\frac{c}{R} + t_{\sigma, id}) (x - \xi)^2)\bigg) - x \bigg| \\
  	& = \bigg|\frac{R}{\sigma'(t_{\sigma, id})} \cdot \bigg(\sigma(t_{\sigma, id}) - \sigma(t_{\sigma, id}) + \frac{x}{R} \sigma'(t_{\sigma, id}) + \frac{x^2}{2R^2} \sigma''(\frac{c}{R} + t_{\sigma, id})\bigg) - x\bigg| \\
  	& = \bigg|\frac{R}{\sigma'(t_{\sigma, id})} \cdot \bigg(\frac{x}{R} \sigma'(t_{\sigma, id}) + \frac{x^2}{2R^2}\sigma''(u)\bigg) - x\bigg| \\
  	& = \bigg| \frac{\sigma''(u) \cdot x^2}{2R \cdot \sigma'(t_{\sigma, id})} + x - x\big| \\
  	& \leq \frac{\| \sigma'' \|_{\infty} \cdot a^2}{2 \cdot |\sigma'(t_{\sigma, id})|} \cdot \frac{1}{R},  
  	\end{split}
  	\end{equation}
  	Wobei sich die letzte Ungleichung aus den Eigenschaften der Supremumsnorm ergibt und zudem aus $x \in [-a,a] \Leftrightarrow -a \leq x \leq a$ durch Quadrieren der Ungleichung folgt, dass $x^2 \leq a^2$ ist.
 	\end{itemize}
\end{proof}