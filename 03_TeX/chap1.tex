\chapter{Grundlagen und Hilfsresultate}
\label{chap:1}

Der Zweck dieses Kapitels ist es, grundlegende Definitionen zu sammeln, die in den folgenden Kapiteln verwendet werden. Weiterhin werden wir Hilfsresultate darstellen und beweisen welche wir vor allem für das Resultat der Konvergenzgeschwindigkeit des einfach berechenbaren Neuronale Netze Regressionsschätzer benötigt werden.

In dieser Arbeit behandeln wir Neuronale-Netze-Regressionsschätzer im Kontext der nichtparametrischen Regression mit zufälligem Design. Im Gegensatz zur parametrischen Regression ist bei der nichtparametrischen, die Bauart der schätzenden Funktion komplett unbekannt, was von Vorteil hat dass weniger Annahmen getroffen werden müssen, man aber dadurch noch mehr Daten benötigt um eine Funktion zu schätzen.

Bei der nichtparametrischen Regressionsschätzung ist seien $(X, Y), (X_1, Y_1), (X_2, Y_2), \dots$ u.i.v $\R^d \times \R$-wertige Zufallsvariablen mit $\E[Y^2] < \infty$. Zudem sei $m\colon \R^d \to R$ definiert durch $m(x) = \E[Y \mid X = x]$ die zugehörige Regressionsfunktion. Ausgehend von 
$$ (X_1, Y_1),\dots,(X_n, Y_n)$$ 
soll $m$ geschätzt werden.

Das Problem der Regressionsschätzung bei zufälligem Design lässt sich wie gefolgt erläutern. In Anwerndung ist üblicherweise die Verteilung von $(X, Y)$ unbekannt, daher kann $m(x) = \E[Y \mid X = x]$ nicht berechnet werden. Oft ist es aber möglich, Werte von $(X, Y)$ zu beobachten. Ziel ist es dann, daraus die Regressionsfunktion zu schätzen. Im Hinblick auf die Minimierung des $L_2$-Risikos sollte dabei der $L_2$-Fehler der Schätzfunktion möglichst klein sein. 

Für das $L_2$-Risiko einer beliebigen messbaren Funktion $f\colon \R^d \to \R$  gilt$\colon$
$$\E[|f(X) - Y|^2] = \E[|m(X) - Y|^2] + \int_{\R^d}|f(x) - m(x)|^2 \mathds{P}_X (dx),$$
d.h. der mittlere quadratische Vorhersagefehler einer Funktion ist darstellbar als Summe des $L_2$-Risikos der Regressionsfunktion (unvermeidbarer Fehler) und des $L_2$-Fehlers der entsteht aufgrund der Verwendung von $f$ anstelle von $m$ bei der Vorhersage bzw. Approximation des Wertes von Y.

Formal führt das daher auf folgende Problemstellung$\colon$
$(X, Y), (X_1, Y_1), (X_2, Y_2), \dots$ seien unabhängig identisch verteilte $\R^d \times \R$ wertige Zufallsvariablen mit $\E[Y^2] < \infty$ und $m\colon\R^d \to \R$ definiert durch $m(x) = \E[Y \mid X = x]$ sei die zugehörige Regressionsfunktion. Gegeben ist die Datenmenge 
$$ \mathfrak{D}_n = \{(X_1, Y_1),\dots,(X_n, Y_n)\}.$$
Gesucht ist eine Schätzung 
$$m_n(\cdot) = m_n(\cdot, \mathfrak{D}_n)\colon\R^d \to \R $$
von $m$, für die der $L_2$-Fehler 
$$\int |m_n(x) - m(x)|^2 \mathds{P}_X(dx)$$
möglichst \glqq klein\grqq ist. (Referenz Györfi (2002))

\section{Definitionen}
Es ist bekannt, dass man Glattheitsvoraussetzungen an die Regressionsfunktion haben muss um nichttriviale Konvergenzresultate für nichtparametrische Regressionsschätzer herzuleiten. Dafür verwenden wir die folgende Definition.
\begin{defn}[($p,C$)-Glattheit]
\label{def:pc}
   Sei $p = q + s$ mit $q \in \N_0$ und $s \in (0,1]$ (also $p \in (0, \infty)$ und sei $C > 0$. Eine Funktion $f\colon \R^d \to \R$ heißt ($p, C$)-glatt, falls für alle $\alpha = (\alpha_1,\dots,\alpha_d) \in \N_0^d$ mit $\sum_{j = 1}^{d}\alpha_j = q$ die partielle Ableitung 
   $$ \frac{\partial^qf}{\partial x_1^{\alpha_1}\dots\partial x_d^{\alpha_d}}$$
   existiert und falls für alle $x, z \in \R^d$ gilt
   $$ \bigg|\frac{\partial^qf}{\partial x_1^{\alpha_1}\dots\partial x_d^{\alpha_d}}(x) - \bigg|\frac{\partial^qf}{\partial x_1^{\alpha_1}\dots\partial x_d^{\alpha_d}}(z) \bigg| \leq C \cdot \|x - z\|^r,$$
   wobei $\|\cdot\|$ die euklidische Norm ist.  
\end{defn}
\begin{bemnumber}
Im Falle von $p \leq 1$ ist keine Funktion ($p, C$)-glatt genau dann, wenn sie Hölder-stetig ist mit Exponent $p$ und Hölder-Konstante $C$.
\end{bemnumber}

Der Ausgangspunkt für die Definition eines neuronalen Netzes ist die Wahl einer Aktivierungsfunktion $\sigma\colon \R \to \R$. Wir haben uns in dieser Arbeit für die sogenannten \glqq squashing functions\grqq, welche eine monoton wachsend ist und für die $\lim_{x \to -\infty}\sigma(x) = 0$ und $\lim_{x \to \infty}\sigma(x) = 1$ gilt. Ein Beispiel für eine squashing function ist der sogenannte sigmoidal bzw. logisitsche squasher
\begin{align}
\label{logsquasher}
\sigma(x) = \frac{1}{1 + \exp(-x)} \quad (x \in \R).
\end{align}

TBD

\begin{defn}
\label{nzulässig}
Sei $n \in \N_0$. Eine Funktion $\sigma\colon \R \to [0, 1]$ wird $N$-zulässig genannt, wenn monoton wachsend und Lipschitz stetig (REFERENZ) ist und wenn zusätzlich die folgenden drei Bedingungen erfüllt sind$\colon$
\begin{itemize}
\item[(i)] Die Funktion $\sigma$ ist $N + 1$ mal stetig differenzierbar mit beschränkten Ableitungen.
\item[(ii)] Es existiert ein Punkt $t_{\sigma} \in \R$, in welchem alle Ableitungen bis hin zur $N$-ten Ableitung von $\sigma$ ungleich Null sind.
\item[(iii)] Wenn $y > 0$ ist, gilt $|\sigma(y) - 1| \leq \frac{1}{y}$. Wenn $y < 0$ ist, gilt $|\sigma(y)| \leq \frac{1}{|y|}$.
\end{itemize}  
\end{defn}  

In Lemma \ref{lem:logsquasher} werden wir zudem zeigen, dass der logistische squasher (\ref{logsquasher}) $N$-zulässig ist für beliebiges $N \in \N.$ 

\section{Hilfsresultate}
\begin{lem}
\label{lem:logsquasher}
Sei $N \in \N$ beliebig, dann erfüllt der logistische squasher $\sigma\colon \R \to [0, 1], \sigma(x) = \frac{1}{1 + \exp(-x)}$ die Bedingungen aus Definition \ref{logsquasher}.  
\end{lem}
\begin{proof}
Sei $N \in \N$ beliebig. Wir wissen, dass $\sigma$ monoton wachsend ist, da für beliebige $s, t \in \R$ mit $s \leq t$ gilt:
$$\sigma(s) = \frac{1}{1 + \exp(-s)} \leq \frac{1}{1 + \exp(-t)} = \sigma(t),$$
wobei wir bei der Ungleichung die Monotonie der Exponentialfunktion verwendet haben und die obige Ungleichung aus
\begin{equation*}
\begin{split}
& \quad \exp(s) \leq \exp(t) \\
 \Leftrightarrow & \quad \exp(-s) \geq \exp(-t) \\
 \Leftrightarrow & \quad 1 + \exp(-s) \geq 1 + \exp(-t) \\
 \Leftrightarrow & \quad \frac{1}{1 + \exp(-s)} \leq \frac{1}{1 + \exp(-t)} \\
\end{split}
\end{equation*}
folgt. Zudem ist $\sigma$ als Komposition $N + 1$ mal stetig differenzierbarer Funktionen selber auch $N + 1$ mal stetig differenzierbar. Die Ableitungen von $\sigma$ haben die Form$\colon$
\begin{equation*}
\begin{split}
\frac{\partial \sigma}{\partial x}(x) &= -\frac{1}{(1 + \exp(-x))^2} \cdot (-\exp(-x)) \\
& = \frac{\exp(-x)}{1 + \exp(-x)} \cdot \frac{1}{1 + \exp(-x)} \\
& = \bigg(1 - \frac{1}{1 + \exp(-x)}\bigg) \cdot \frac{1}{1 + \exp(-x)} \\
& = (1 - \sigma(x)) \cdot \sigma(x).
\end{split}
\end{equation*}
Da wir bei weiterem Ableiten die Produktregel wiederholt anwenden sind alle Ableitungen von $\sigma$, Polynome in $\sigma$. Dadurch folgt Bedingung (i) aus Definition \ref{logsquasher}, da $sigma$ nach Voraussetzung durch $0$ und $1$ beschränkt ist, und die Ableitungen von $\sigma$ als Produkt von beschränkten Faktoren daher auch. Da hiermit auch die erste Ableitung von $\sigma$ beschränkt ist wissen wir nach Satz ... aus (REFERENZ), dass $\sigma$ Lipschitz stetig ist.
Nun kommen wir zum Beweis von Bedingung (ii). Polynome, die nicht das $0$-Polynom sind, haben nach Satz ... (REFERENZ) auf $(0, 1)$ endlich viele Nullstellen und $\sigma$ bildet nach Voraussetzung in das Intervall $[0, 1] \supseteq (0, 1)$ ab. Da die Ableitungen von $\sigma$, als Zusammensetzung von Polynome in $\sigma$, wieder Polynome sind für die die obere Eigenschaft ebenfalls gilt, existiert ein $t_{\sigma} \in \R$ mit$\sigma(t_{\sigma}) \neq 0$ sodass alle Ableitungen bis zum Grad $N$ von $\sigma$, aufgrund ihrer Struktur ungleich $0$ sind. Daher ist Bedingung (ii) ebenfalls erfüllt.
Betrachten wir nun ein beliebiges $x > 0$. Dann wissen wir nach dem Mittelwertsatz (REFERENZ) dass ein $z \in (0, \infty)$ existiert, sodass mit $\exp(z) > 1$, da $z > 0$ ist, gilt$\colon$
\begin{equation*}
\begin{split}
(\exp(x - 1) \cdot (x - 0) = e^z
\end{split}
\end{equation*} 
und da $x \neq 0$ ist, daraus folgt, dass $$\frac{\exp(x) - 1}{x - 0} > 1$$ gilt und damit dann auch durch Multiplikation mit $x$ insbesondere $$ x \leq \exp(x) + 1.$$ Daraus erhalten wir mit Umformungen da $x > 0$ und $1 + \exp(-x) > 0$ ist$\colon$
\begin{equation*}
\begin{split}
& \quad x \leq \exp(x) + 1 \\
 \Leftrightarrow & \quad x \cdot \exp(-x) \leq 1 + \exp(-x) \\
 \Leftrightarrow & \quad \frac{\exp(-x)}{1 + \exp(-x)} \leq \frac{1}{x} \\
 \Leftrightarrow & \quad 1 - \frac{1}{1 + \exp(-x)} \leq \frac{1}{x} \\
 \Leftrightarrow & \quad |\sigma(x) - 1| \leq \frac{1}{x}.
\end{split}
\end{equation*}
Wobei die letzte Ungleichung aus der Eigenschaft des Betrags kommt, da $\frac{1}{1 + \exp(-x)} - 1< 0$ ist, weil $1 + \exp(-x) > 1$, da $\exp(-x) > 0$. Dies zeigt die erste Relation aus Bedingung (iii).
Die zweite Relation folgt durch die gleiche Art und Weise, da wir durch $$\frac{1}{1 + \exp(x)} - \frac{1}{2}= \sigma(0 - x) - \frac{1}{2} = -\sigma(0 + x) + \frac{1}{2} = -\frac{1}{1 + \exp(-x)} + \frac{1}{2}$$ 
wissen, dass $\sigma$ punktsymmetrisch in $(0, \frac{1}{2})$ ist. Die obige Gleichheit folgt aus
\begin{equation*}
\begin{split}
 & \quad \frac{1}{1 + \exp(x)} - \frac{1}{2} = -\frac{1}{1 + \exp(-x)} + \frac{1}{2} \\
 \Leftrightarrow  & \quad \frac{1}{1 + \exp(x)} + \frac{1}{2} -\frac{1}{1 + \exp(-x)} = \frac{1}{2}\\
 \Leftrightarrow & \quad  \frac{1 + \exp(-x) + 1 + \exp(x)}{(1 + \exp(x)) \cdot (1 + \exp(-x))} - \frac{1}{2} = \frac{1}{2} \\
 \Leftrightarrow & \quad \frac{2 + \exp(-x) + \exp(x)}{2 + \exp(-x) + \exp(x)} - \frac{1}{2} = \frac{1}{2} \\
 \Leftrightarrow & \quad 1 - \frac{1}{2} = \frac{1}{2}.
\end{split}
\end{equation*}
Aus dieser Eigenschaft folgt mit $$\sigma(- x) - 1 = \frac{1}{1 + \exp(x)} - 1 = -\frac{1}{1 + \exp(-x)} = -\sigma(x)$$ für $x < 0$ aus der ersten Relation, da $-x > 0 $ ist$\colon$  
$$|\sigma(x)| = |-\sigma(x)| = |\sigma(- x) - 1| \leq \frac{1}{-x} \leq \frac{1}{|x|}.$$
Damit haben wir alle drei Bedingungen aus Definition \ref{logsquasher} gezeigt und unsere Aussage bewiesen.
\end{proof}
\begin{lem}
  \label{lem:1}
  Sei $\sigma \colon \R \to \R$ eine Funktion und $R$, $a > 0$.
  \begin{itemize}
  \item[a)] Angenommen $\sigma$ ist zwei mal stetig differenzierbar und $t_{\sigma,id} \in \R$ so, dass $sigma'(t_{\sigma, id}) \neq 0$ ist. Dann gilt mit
  $$ f_{id}(x) = \frac{R}{\sigma'(t_{\sigma, id})} \cdot \left(\sigma\left(\frac{x}{R} + t_{\sigma, id}\right) - \sigma(t_{\sigma, id})\right)$$
  für beliebige $x \in [-a, a]\colon$ 
  $$ |f_{id}(x) - x| \leq \frac{\|\sigma''\|_{\infty} \cdot a^2}{2 \cdot |\sigma'(t_{\sigma, id})|} \cdot \frac{1}{R}.$$
  \item[b)] Angenommen $\sigma$ ist drei mal stetig differenzierbar und $t_{\sigma,sq} \in \R$ so, dass $sigma''(t_{\sigma, sq}) \neq 0$ ist. Dann gilt mit
  $$ f_{sq}(x) = \frac{R^2}{\sigma''(t_{\sigma, sq})} \cdot \left(\sigma\left(\frac{2 \cdot x}{R} + t_{\sigma, sq}\right) - 2 \cdot \sigma(\frac{x}{R} + t_{\sigma, sq})+ \sigma(t_{\sigma, sq})\right)$$
  für beliebige $x \in [-a, a]\colon$ 
  $$ |f_{sq}(x) - x^2| \leq \frac{5 \cdot \|\sigma'''\|_{\infty} \cdot a^3}{3 \cdot |\sigma'(t_{\sigma, sq})|} \cdot \frac{1}{R}.$$
  \end{itemize}
\end{lem}
\begin{proof}
	\begin{itemize}
  	\item[a)] Sei $u = \frac{c}{R} + t_{\sigma, id}$, $\xi = 0$ und  $x \in [-a, a]$beliebig. Wir wissen, dass $f_{id}$ 1-mal differenzierbar ist, da nach Vorraussetzung $\sigma$ 2-mal stetig differenzierbar ist,  existiert nach der Restgliedformel von Lagrange (REFERENZ) ein $c \in [\xi, x] $, sodass mit Ausklammern von $\frac{R}{\sigma'(t_{\sigma, id})} $ folgt$\colon
$

  	\begin{equation*}
  	\begin{split}
  	 |f_{id}(x) -  & x| \\
  	& \leq \bigg|\frac{R}{\sigma'(t_{\sigma, id})} \cdot \bigg(\sigma\left(\frac{\xi}{R} + t_{\sigma, id}\right) - \sigma(t_{\sigma, id}) + \frac{1}{R} \sigma'\left(\frac{\xi}{R} + t_{\sigma, id}\right) (x - \xi) \\ & \qquad + \frac{1}{2R^2} \sigma''(\frac{c}{R} + t_{\sigma, id}) (x - \xi)^2)\bigg) - x \bigg| \\
  	& = \bigg|\frac{R}{\sigma'(t_{\sigma, id})} \cdot \bigg(\sigma(t_{\sigma, id}) - \sigma(t_{\sigma, id}) + \frac{x}{R} \sigma'(t_{\sigma, id}) + \frac{x^2}{2R^2} \sigma''(\frac{c}{R} + t_{\sigma, id})\bigg) - x\bigg| \\
  	& = \bigg|\frac{R}{\sigma'(t_{\sigma, id})} \cdot \bigg(\frac{x}{R} \sigma'(t_{\sigma, id}) + \frac{x^2}{2R^2}\sigma''(u)\bigg) - x\bigg| \\
  	& = \bigg| \frac{\sigma''(u) \cdot x^2}{2R \cdot \sigma'(t_{\sigma, id})} + x - x\big| \\
  	& \leq \frac{\| \sigma'' \|_{\infty} \cdot a^2}{2 \cdot |\sigma'(t_{\sigma, id})|} \cdot \frac{1}{R},  
  	\end{split}
  	\end{equation*}
  	Wobei sich die letzte Ungleichung aus den Eigenschaften der Supremumsnorm ergibt und zudem aus $x \in [-a,a] \Leftrightarrow -a \leq x \leq a$ durch Quadrieren der Ungleichung folgt, dass $x^2 \leq a^2$ ist.
  	\item[b)] Folgt analog wie in a) durch 2-maliges Anwenden der Restgliedformel von Lagrange (REFeRENZ) auf die Funktion $f$ die hier nun 2-mal differenzierbar ist, da $\sigma$ nach Voraussetzung  3-mal stetig differenzierbar ist.
 	\end{itemize}
\end{proof}

\begin{lem}
  \label{lem:2}
  Sei $\sigma \colon \R \to [0, 1]$ nach Definition 1, 2-zulässig. Zudem sei $R > 0$ und $a > 0$ beliebig. Dann gilt für das neuronale Netz
  \begin{equation*}
  	\begin{split}
  	f_{mult}(x, y) = \frac{R^2}{4 \cdot \sigma''(t_{\sigma})} \cdot & \bigg(\sigma\Big(\frac{2 \cdot (x + y)}{R} + t_{\sigma} \Big) - 2 \cdot \sigma\Big(\frac{x + y}{R} + t_{\sigma}\Big) \\
  	& - \sigma \Big(\frac{2 \cdot (x - y)}{R} + t_{\sigma} \Big) + 2 \cdot \sigma \Big(\frac{x - y}{R} + t_{\sigma} \Big) \bigg)
  	\end{split}
  	\end{equation*}
  	für beliebige $x, y \in [-a, a]$ die folgende Ungleichung$\colon$
  	$$|f_{mult}(x, y) - x \cdot y| \leq \frac{20 \cdot \|\sigma'''\|_{\infty} \cdot a ^3}{3 \cdot |\sigma''(t_{\sigma})|} \cdot \frac{1}{R}.$$
  \end{lem}
  \begin{proof}
  Durch Ausmultiplizieren erhalten wir $$f_{mult}(x,y) = \frac{1}{4}(f_{sq}(x + y) - f_{sq}(x - y))$$ und $$x \cdot y = \frac{1}{4}\big((x + y)^2 - (x - y)^2\big).$$
  Aus diesen beiden Gleichungen folgt durch Ausklammern von $\frac{1}{4}$, der Homogenität des Betrags und der Anwendung der Dreickecksungleichung$\colon$
  \begin{equation*}
  \begin{split}
  |f_{mult}(x, y) - x \cdot y| & = \frac{1}{4} \cdot \big|f_{sq}(x + y) - f_{sq}(x - y) - (x + y)^2 + (x - y)^2\big| \\
  & \leq \frac{1}{4} \cdot \big|f_{sq}(x + y) - (x + y)^2\big| + \frac{1}{4}\cdot\big| (x - y)^2 - f_{sq}(x - y)\big| \\
  & \leq 2 \cdot \frac{1}{4} \cdot \frac{40 \cdot \|\sigma'''\|_{\infty} \cdot a^3}{3 \cdot |\sigma'(t_{\sigma, sq})|} \cdot \frac{1}{R} \\
  & = \frac{20 \cdot \|\sigma'''\|_{\infty} \cdot a ^3}{3 \cdot |\sigma''(t_{\sigma})|} \cdot \frac{1}{R},
  \end{split}
\end{equation*}   
wobei bei bei der letzten Ungleichung verwendet haben, dass $a > 0$ nach Lemma \ref{lem:1}b) beliebig gewählt wurde und daher insbesondere für beliebiges $x \in [-2a,2a]$ $$ |f_{sq}(x) - x^2| \leq \frac{40 \cdot \|\sigma'''\|_{\infty} \cdot a^3}{3 \cdot |\sigma'(t_{\sigma, sq})|} \cdot \frac{1}{R}$$ gilt. 
  \end{proof}
  \begin{lem}
  \label{lem:3}
  Sei $\sigma\colon \R \to [0, 1]$ nach Definition 1, 2-zulässig. Sei $f_{mult}$ das neuronale Netz aus Lemma \ref{lem:2} und $f_{id}$ das neuronale Netz aus Lemma \ref{lem:1}. Angenommen es gilt $$a \geq 1 \text{\quad und \quad} R \geq \frac{\|\sigma''\|_{\infty} \cdot a}{2 \cdot |\sigma'(t_{\sigma, id})|}.$$ 
  Dann erfüllt das neuronale Netz 
  \begin{equation*}
  \begin{split}
  f_{ReLU}(x) & = f_{mult}(f_{id}(x), \sigma(R \cdot x)) 
  \end{split}
  \end{equation*}
 für alle $x \in [-a, a]\colon$
 $$|f_{ReLU}(x) - \max\{x, 0\}| \leq 56 \cdot \frac{\max\{|\sigma''\|_{\infty}, \|\sigma'''\|_{\infty}, 1\}}{\min\{2 \cdot |\sigma'(t_{\sigma, id}), |\sigma''(t_{\sigma})|, 1\}} \cdot a^3 \cdot \frac{1}{R}.$$
  \end{lem}
  \begin{proof}
  Da $\sigma$ nach Voraussetzung 2-zulässig nach Definition 1 ist, gilt für $R \geq 0,$ und $x \in \R\setminus\{0\}\colon$ $$|\sigma(R \cdot x) - 1| \leq \frac{1}{R\cdot x} \text{\quad für \quad} x > 0$$ und $$|\sigma(R \cdot x)| \leq \frac{1}{|R \cdot x|} \text{\quad für \quad} x < 0.$$
Damit folgt aus der Homogenität des Betrags $$|\sigma(R \cdot x) - \1_{[0, \infty)}(x)| \leq \frac{1}{|R \cdot x|} = \frac{1}{R \cdot |x|}.$$ 
Nach Lemma \ref{lem:1} und Lemma \ref{lem:2} gilt$\colon$
$$ |f_{id}(x) - x| \leq \frac{\|\sigma''\|_{\infty} \cdot a^2}{2 \cdot |\sigma'(t_{\sigma, id})|} \cdot \frac{1}{R} \text{\quad für \quad} x \in [-a, a]$$ und 
$$ |f_{mult}(x, y) - x \cdot y| \leq \frac{160 \cdot \|\sigma'''\|_{\infty} \cdot a ^3}{3 \cdot |\sigma''(t_{\sigma})|} \cdot \frac{1}{R} \text{\quad für \quad} x \in [-2a, 2a].$$ 
Da nach Voraussetzung $a \geq 1$ ist gilt insbesondere $[0, 1] \in [-2a, 2a]$ und daher gilt insbesondere $\sigma(x) \in [-2a, 2a].$ 
Zudem erhalten wir durch eine Nulladdition, das Anwenden der Dreiecksungleichung, die Verwendung von Lemma \ref{lem:1} und der Voraussetzung für $R\colon$
\begin{equation*}
\begin{split}
|f_{id}(x)| & = |f_{id}(x) - x + x| \\
& = |f_{id}(x) -x| + |x| \\
& \leq |f_{id}(x) - x| \leq \frac{\|\sigma''\|_{\infty} \cdot a^2}{2 \cdot |\sigma'(t_{\sigma, id})|} \cdot \frac{1}{R} + |x| \\
& \leq |f_{id}(x) - x| \leq \frac{\|\sigma''\|_{\infty} \cdot a^2}{2 \cdot |\sigma'(t_{\sigma, id})|} \cdot \frac{2 \cdot |\sigma'(t_{\sigma, id})|}{\|\sigma''\|_{\infty} \cdot a} + |x| \\
& = a + |x| \\
& = 2 \cdot a
\end{split}
\end{equation*}
wobei $x \in [-a, a]$. Daraus folgt insbesondere $f_{id}(x) \in [-2a, 2a].$
Mithilfe von $\max\{x, 0 \} = x \cdot \1_{[0, \infty)}(x)$, der Voraussetzung, zweier Nulladdition und dem zweifachen Anwenden der Dreiecksungleichung erhalten wir$\colon$
\begin{equation*}
\begin{split}
|f_{ReLU}(x) - & \max\{x, 0\}|  \\
& = \big| f_{mult}(f_{id}(x), \sigma(R \cdot x)) - x \cdot \1_{[0, \infty)}(x)\big| \\
& \leq \big| f_{mult}(f_{id}(x), \sigma(R \cdot x)) - f_{id}(x)\cdot\sigma(R \cdot x)\big| \\
& \qquad + \big| f_{id}(x)\cdot\sigma(R \cdot x) - x \cdot \sigma(R \cdot x)\big| + \big| x \cdot \sigma(R \cdot x) - x \cdot \1_{[0, \infty)}(x)\big|. \\
& \text{Daraus ergibt sich mithilfe der obigen Eigenschaften und $a^3 \geq 1$}\\
& \leq \frac{160 \cdot \|\sigma'''\|_{\infty} \cdot a ^3}{3 \cdot |\sigma''(t_{\sigma})|} \cdot \frac{1}{R} + \frac{\|\sigma''\|_{\infty} \cdot a^3}{2 \cdot |\sigma'(t_{\sigma, id})|} \cdot \frac{1}{R} \cdot 1+ \frac{1}{R} \\
& \leq \bigg(\frac{160}{3} \cdot \frac{\|\sigma'''\|_{\infty} \cdot a ^3}{|\sigma''(t_{\sigma})|} + \frac{\|\sigma''\|_{\infty} \cdot a^3}{2 \cdot |\sigma'(t_{\sigma, id})|} + \frac{a^3}{a^3} \bigg) \cdot \frac{1}{R} \\ 
& \leq \bigg(\frac{160 \cdot\|\sigma'''\|_{\infty} \cdot a ^3 + 3 \cdot \|\sigma''\|_{\infty} \cdot a^3 + 3 \cdot a^3}{3 \cdot \min\{ 2 \cdot \sigma'(t_{\sigma, id})|, |\sigma''(t_{\sigma})|, 1\}}\bigg) \cdot \frac{1}{R}\\
& \leq \frac{166}{3} \cdot \bigg(\frac{\max\{\|\sigma'''\|_{\infty}, \|\sigma''\|_{\infty} , 1\}}{\min\{ 2 \cdot \sigma'(t_{\sigma, id})|, |\sigma''(t_{\sigma})|, 1\}}\bigg) \cdot a^3 \cdot  \frac{1}{R} \\
& \leq 56 \cdot \frac{\max\{|\sigma''\|_{\infty}, \|\sigma'''\|_{\infty}, 1\}}{\min\{2 \cdot |\sigma'(t_{\sigma, id}), |\sigma''(t_{\sigma})|, 1\}} \cdot a^3 \cdot \frac{1}{R}.
\end{split}
\end{equation*}
  \end{proof}
  \begin{lem}
  \label{lem:4}
  Sei $M \in \N$ und sei $\sigma\colon \R \to [0, 1]$ 2-zulässig nahc Definition .... Sei $a > 0$ und $$R \geq \frac{\|\sigma''\|_{}\infty \cdot (M + 1)}{2 \cdot |\sigma'(t_{\sigma, id})|},$$ sei $y \in [-a, a]$ und $f_{ReLU}$ das neuronale Netz aus Lemma \ref{lem:3}. Dann erfüllt das neuronale Netz 
  \begin{equation*}
  \begin{split}
  f_{hat,y}(x) = f_{ReLU}& \bigg(\frac{M}{2a} \cdot (x - y) + 1\bigg) - 2 \cdot f_{ReLU}\bigg(\frac{M}{2a} \cdot (x - y)\bigg) \\ &+ f_{ReLU}\bigg(\frac{M}{2a} \cdot (x - y) - 1\bigg)
  \end{split}
  \end{equation*}
  für alle $x \in [-a ,a]\colon$ 
  $$\bigg|f_{hat,y}(x) - \bigg(1 - \frac{M}{2a} \cdot |x - y|\bigg)_+\bigg| \leq 1792 \cdot \frac{\max\{|\sigma''\|_{\infty}, \|\sigma'''\|_{\infty}, 1\}}{\min\{2 \cdot |\sigma'(t_{\sigma, id}), |\sigma''(t_{\sigma})|, 1\}} \cdot M^3 \cdot \frac{1}{R}.$$
  \end{lem}
  \begin{proof}
  Als erstes zeigen wir $$(1 - \frac{M}{2a} \cdot |x|)_+ = \max\{\frac{M}{2a} \cdot x + 1, 0\} - 2 \cdot \max\{\frac{M}{2a} \cdot x, 0\} + \max\{\frac{M}{2a} \cdot x - 1, 0\}, \qquad (x \in \R)$$ damit wir das Resultat mithilfe von Lemma \ref{lem:3} beweisen können.
  Um die obige Gleichung zu zeigen unterscheiden wir vier Fälle.
  \begin{itemize}
  \item[Fall 1] ($x < - \frac{M}{2a}$) In diesem Fall hat die linke Seite durch $z_+ = \max\{z, 0\} \quad (z \in \R)$ und nach der Definition des Betrags die Gestalt $$\max\{1 + \frac{M}{2a} \cdot x, 0\}$$ und die rechte Seite die Form $$\max\{\frac{M}{2a} \cdot x + 1, 0\} - 2 \cdot 0 + 0,$$ da $x < 0$ und damit die letzten zwei Summanden 0 sind. Damit sind stimmt die rechte Seite mit der linken überein. $\hfill(\square)$
  \item[Fall 2] ($ - \frac{M}{2a} \leq x \leq 0$) Dieser Fall liefert aufgrund der nicht Positivität von $x$ analog das selbe Resultat wie Fall 1. $\hfill(\square)$
  \item[Fall 3] ($0 < x \leq \frac{M}{2a}$) In diesem Fall hat die linke Seite nach der Definition des Betrags die Gestalt $$\max\{1 - \frac{M}{2a} \cdot x, 0\}$$ und die rechte Seite die Form $$\max\{\frac{M}{2a} \cdot x + 1, 0\} - 2 \cdot \max\{\frac{M}{2a} \cdot x, 0\} + \max\{\frac{M}{2a} \cdot x - 1, 0\},$$ und erfordert daher eine weitere Fallunterscheidung.
  \item[Fall 3.1] ($x \cdot \frac{M}{2a} \leq 1$) In diesem Fall gilt für die linke Seite$\colon$ 
  $$\max\{1 - \frac{M}{2a} \cdot x, 0\} = 1 - \frac{M}{2a} \cdot x$$ und für die rechte Seite$\colon$ 
  $$\max\{\frac{M}{2a} \cdot x + 1, 0\} - 2 \cdot \max\{\frac{M}{2a} \cdot x, 0\} + 0 = \frac{M}{2a} \cdot x + 1 - 2 \cdot \frac{M}{2a} \cdot x = 1 - \frac{M}{2a} \cdot x,$$ und stimmt daher mit der linken Seite überein.
  \item[Fall 3.2] ($x \cdot \frac{M}{2a} > 1$) In diesem Fall gilt für die linke Seite$\colon$ 
  $$\max\{1 - \frac{M}{2a} \cdot x, 0\} = 0$$ und für die rechte Seite$\colon$ 
  $$\frac{M}{2a} \cdot x + 1 - 2 \cdot \frac{M}{2a} \cdot x + \frac{M}{2a} \cdot x - 1 = 0$$ und stimmt daher mit der linken Seite überein. Damit ist Fall 3 gezeigt. $\hfill(\square)$
  \item[Fall 4] ($\frac{M}{2a} < x$)  In diesem Fall hat die linke Seite nach der Definition des Betrags die Gestalt $$\max\{1 - \frac{M}{2a} \cdot x, 0\}$$ und die rechte Seite die Form $$\max\{\frac{M}{2a} \cdot x + 1, 0\} - 2 \cdot \max\{\frac{M}{2a} \cdot x, 0\} + \max\{\frac{M}{2a} \cdot x - 1, 0\},$$ und erfordert daher eine weitere Fallunterscheidung.
  \item[Fall 4.1] ($\frac{M}{2a} \cdot x \leq 1$) In diesem Fall gilt für die linke Seite$\colon$ 
  $$\max\{1 - \frac{M}{2a} \cdot x, 0\} = 1 - \frac{M}{2a} \cdot x$$ und für die rechte Seite$\colon$ 
  $$\frac{M}{2a} \cdot x + 1 - 2 \cdot \frac{M}{2a} \cdot x + 0 = 1 - \frac{M}{2a}$$ und stimmt daher mit der linken Seite überein.
  \item[Fall 4.2] ($\frac{M}{2a} \cdot x > 1$) In diesem Fall gilt für die linke Seite$\colon$ 
  $$\max\{1 - \frac{M}{2a} \cdot x, 0\} = 0$$ und für die rechte Seite$\colon$ 
  $$\frac{M}{2a} \cdot x + 1 - 2 \cdot \frac{M}{2a} \cdot x + \frac{M}{2a} \cdot x - 1 = 0$$ und stimmt daher mit der linken Seite überein. Damit ist Fall 4 gezeigt. $\hfill(\square)$
\end{itemize} 
Durch diese Fallunterscheidung wurde die obige Gleichung (REFERENZ) bewiesen. Daraus folgt mit der Definition von $f_{hat,y}(x)$ und zwei mal der Dreiecksungleichung
\begin{equation*}
\begin{split}
\bigg|f_{hat,y}(x) - \bigg(1 - \frac{M}{2a} \cdot |x - y|\bigg)_+\bigg| \leq \bigg|&f_{ReLU} \bigg(\frac{M}{2a} \cdot (x - y) + 1\bigg) - \max\{\frac{M}{2a} \cdot (x - y) + 1, 0\}\bigg| \\ 
& + 2 \cdot \bigg|f_{ReLU}\bigg(\frac{M}{2a} \cdot (x - y)\bigg) - \max\{\frac{M}{2a} \cdot (x - y), 0\}\bigg| \\
& + \bigg|f_{ReLU}\bigg(\frac{M}{2a} \cdot (x - y) - 1\bigg) - \max\{\frac{M}{2a} \cdot (x - y) - 1, 0\}\bigg|\\ 
& \leq 1792 \cdot \frac{\max\{|\sigma''\|_{\infty}, \|\sigma'''\|_{\infty}, 1\}}{\min\{2 \cdot |\sigma'(t_{\sigma, id}), |\sigma''(t_{\sigma})|, 1\}} \cdot M^3 \cdot \frac{1}{R},
\end{split}
\end{equation*} 
wobei die letzte Ungleichung daraus folgt, dass wir auf jeden Summanden mit $1 \leq a = M + 1$ Lemma \ref{lem:3} angewendet haben und die Abschätzung 
$$ (M + 1)^3 = M^3 + 3 \cdot M^2 + 3 \cdot M + 1 \leq M^3 + 3 \cdot M^3 + 3 \cdot M ^3 + M^3 = 8 \cdot M^3 \quad (M \in \N)$$ verwendet haben.
  \end{proof}
  Das nächste Lemma stammt aus Schmidt-Hieber (2019) und wurde auf die Notation dieser Arbeit abgeändert.
  \begin{lem}
  \label{lem:5}
  Sei $M \in \N$, $c_1$ eine Konstante,$a > 0$, $m$ eine ($p, C$)-glatte Funktion und $P(x)$ eine lokale Konvexkombination von Taylorpolynomen in $m$. Dann gilt$\colon$
  $$\sup_{x \in [-a, a]^d} |m(x) - P(x)| \leq c_1 \cdot \frac{1}{M^p}.$$
  \end{lem}