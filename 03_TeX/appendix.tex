\chapter*{Appendix}
\markboth{APPENDIX}{}
\addcontentsline{toc}{chapter}{Appendix}
\label{chap:app}

Der Programmcode ist wie folgt aufgebaut:
%\linespread{1.0}
\begin{itemize}
    \item[-] \texttt{main.py} ist das Hauptprogramm welches alle Schätzer aufruft und die Ouputs generiert.
    \item[-] \texttt{data\_gen.py} generiert die Daten die wir für unsere Simulation benötigen.
    \item[-] \texttt{help\_neural\_networks.py} fasst alle Hilfsfunktionen zusammen.
    \item[-] \texttt{new\_neural\_network.py} enthält unseren Neuronale-Netze-Regressionsschätzer.
    \item[-] \texttt{fc\_neural\_network.py} enthält das neuronale Netz mit einer verborgenen Schicht.
    \item[-] \texttt{nearest\_neighbor.py} enthält einen Nächste-Nachbar Schätzer.
    \item[-] \texttt{constant.py} enthält den konstanten Schätzer.
\end{itemize}
\renewcommand\lstlistingname{Listing}
\lstinputlisting[basicstyle=\scriptsize, caption={\texttt{main.py}},captionpos=t]{../01_Code/main.py}
\lstinputlisting[basicstyle=\scriptsize, caption={\texttt{data\_gen.py}},captionpos=t]{../01_Code/data_gen.py}
\lstinputlisting[basicstyle=\scriptsize, caption={\texttt{help\_neural\_networks.py}},captionpos=t]{../01_Code/help_neural_networks.py}
\lstinputlisting[basicstyle=\scriptsize, caption={\texttt{new\_neural\_network.py}},captionpos=t]{../01_Code/new_neural_network.py}
\lstinputlisting[basicstyle=\scriptsize, caption={\texttt{fc\_neural\_network.py}},captionpos=t]{../01_Code/fc_neural_network.py}
\lstinputlisting[basicstyle=\scriptsize, caption={\texttt{nearest\_neighbor.py}},captionpos=t]{../01_Code/nearest_neighbor.py}
\lstinputlisting[basicstyle=\scriptsize, caption={\texttt{constant.py}},captionpos=t]{../01_Code/constant.py}