\chapter{Resultat zur Konvergenzgeschwindigkeit}
\label{chap:3}

In diesem Kapitel stellen wir das Hauptresultat dieser Arbeit vor.
Wir betrachten im Folgenden das Problem~{\ref{prblm:1}} und wollen nun eine Abschätzung des erwarteten $L_2$-Fehlers 
$$\E \Big[\int |m_n(x) - m(x)|^2  \mathds{P}_X(dx)\Big]$$
im Falle des Schätzers 
\begin{equation}
\label{schätzer}
m_n(x) \coloneqq T_{\beta_n}\tilde{m}_n (x),
\end{equation}
mit $\beta_n = c_1 \cdot \log(n)$ für eine hinreichend große und von $n$ unabhängige Konstante $c_1 > 0$ und unter Annahme einer $(p,C)$-glatten Regressionsfunktion $m$ herleiten. Hierbei bezeichnet $T_{\beta}z = \max\{\min\{z, \beta\}, -\beta\}$
für $z \in \R$ und $\beta > 0$. Weiterhin bezeichnen wir mit $\tilde{m}_n$ unseren Neuronale-Netze-Regresssionsschätzer aus Kapitel~\ref{subsec:2.2}, welcher aus mehreren neuronalen Netzen konstruiert wurde. Für diesen gilt: 
Die Aktivierungsfunktion $\sigma$ ist der logistische Squasher, $N \geq q$, $M = M_n = \lceil c_2 \cdot n^{1/(2p + d)}\rceil$ mit $c_2 >0$ und unabhängig von $n$, $R = R_n = n^{d + 4}$ und $a = a_n = (\log n)^{1/(6(N + d))}.$

Nun kommen wir zu dem Hauptresultat dieser Arbeit.
\begin{mthm}[{\cite[Theorem 1]{kohler19}}]
\label{optstop}
Angenommen die Verteilung von $Y$ erfüllt 
$$ \E\Big[\mathrm{e}^{c_3 \cdot |Y|^2}\Big] < \infty$$
für eine Konstante $c_3 > 0$ und die Verteilung von $X$ besitzt einen beschränkten Träger $\supp(\mathds{P}_X)$. Sei $m(x) = \E[Y \mid X = x]$ die zu dem Tupel $(X, Y)$ gehörige Regressionsfunktion.
Angenommen $m$ ist $(p,C)$-glatt, mit $p = q + s$, wobei $q \in \N_0$, $s \in (0,1]$ und $C > 0$ ist. Des Weiteren sei $m_n$ der in Gleichung~(\ref{schätzer}) definierte Regressionsschätzer.

Dann gilt für hinreichend großes $n$:
$$\E \Big[\int |m_n(x) - m(x)|^2  \mathds{P}_X(dx)\Big] \leq c_{\mathrm{fin}} \cdot (\log n)^3 \cdot n^{- \frac{2p}{2p + d}},$$
wobei $c_{\mathrm{fin}} > 0$ eine von $n$ unabhängige Konstante ist.
\end{mthm}

Bevor wir zum Beweis von Hauptresultat ~\ref{optstop} kommen, stellen wir im folgenden Abschnitt die dafür notwendigen Hilfsresultate vor.
\section{Approximationsresultate für Hauptsatz \hyperref[optstop]{3.1}}
Die nächsten  Definitionen und Lemmata benötigen wir für den Beweis unseres Hauptresultats, einer Aussage über die Konvergenzgeschwindigkeit unseres Neuronale-Netze-Regressionsschätzers. Die Lemmata werden hier nur der Vollständigkeit halber und ohne Beweis aufgeführt. 
Als Nächstes geben wir eine Definition von Überdeckungszahlen an, da wir im Beweis für unser Hauptresultat eine Abschätzung einer $L_p\text{-}\epsilon$-Überdeckungszahl anwenden.
\begin{defn}
\label{ueberdeckung}
Sei $(X, \delta)$ ein pseudometrischer Raum (vgl.\@ \cite[Definition~2.1.1]{Topologie2015}). Für $x \in X$ und $\epsilon > 0$ sei:
$$U_{\epsilon}(x) = \{z \in X : \delta(x, z) < \epsilon\}$$
die offene Kugel um $x$ mit Radius $\epsilon$.
\begin{itemize}
\item[a)] $\{z_1,\dots,z_N\} \subseteq X$ heißt $\epsilon$\textit{-Überdeckung} einer Menge $A \subseteq X$, falls gilt:
$$A \subseteq \bigcup_{k = 1}^N U_{\epsilon}(z_k).$$
\item[b)] Ist $A \subseteq X$ und $\epsilon > 0$, so ist die sogenannte $\epsilon$\textit{-Überdeckungszahl} von $A$ in $(X,\delta)$ definiert als:
$$\mathcal{N}_{(X,\delta)}(\epsilon, A) \coloneqq \inf\big\{|U| : U \subseteq X \text{ ist $\epsilon$-Überdeckung von } A\big\}.$$   
\end{itemize}
\end{defn}
Da wir in unserem Hauptresultat eine Überdeckungszahl von Mengen von Funktionen betrachten werden, benötigen wir folgende Definition.
\begin{defn}
\label{lpe}
Sei $\mathcal{F}$ eine Menge von Funktionen $f\colon \R^d \to \R$, sei $\epsilon > 0$, $1 \leq p < \infty$ und seien $x_1,\dots,x_n \in \R^d$ und $x_1^n = (x_1,\dots,x_n).$ Dann ist die $L_p$-$\epsilon$\textit{-Überdeckungszahl} \emph{von $\mathcal{F}$ auf $x_1^n$} definiert durch:
$$\mathcal{N}_p(\epsilon, \mathcal{F}, x_1^n) \coloneqq \mathcal{N}_{(X,\delta)}(\epsilon, \mathcal{F}),$$
wobei der pseudometrische Raum $(X, \delta)$ gegeben ist durch $X$ als Menge aller Funktionen $f\colon \R^d \to \R$ und durch die Pseudometrik $\delta(f, g) = \delta_p(f, g) = (\frac{1}{n}\sum_{i = 1}^n |f(x_i) - g(x_i)|^p)^{1/p}.$
\end{defn}
%In anderen Worten: $\mathcal{N}_p(\epsilon, \mathcal{F}, x_1^n)$ ist das minimale $N \in \N$, so dass Funktionen $f_1,\dots,f_N\colon \R^d \to \R$ existieren mit der Eigenschaft, dass für jedes $f \in \mathcal{F}$ gilt:
%$$\min_{j = 1,\dots,N}\bigg(\frac{1}{n}\sum_{i = 1}^n|f(x_i) - f_j(x_i)|^p\bigg)^{1/p} < \epsilon.$$
Das folgende Lemma beschreibt wie man den erwarteten $L_2$-Fehler eines Schätzers mithilfe einer $L_p$-$\epsilon$\textit{-Überdeckungszahl} abschätzen kann. Dieses Lemma ist ein Spezialfall von \cite[Lemma 8]{kohler19}.
  \begin{lem}
  \label{lem:8}
Sei $\beta_n = c_1 \cdot \log(n)$ für eine hinreichend große Konstante $c_1 > 0$. Angenommen die Verteilung von $Y$ erfüllt 
$$ \E\Big[\mathrm{e}^{c_2 \cdot |Y|^2}\Big] < \infty$$
für eine Konstante $c_2 > 0.$ Zudem nehmen wir an, dass die Regressionsfunktion~$m$ beschränkt ist. Sei $\mathcal{F}_n$ eine Menge von Funktionen $f\colon \R^d \to \R.$ Sei $\hat{m}_n$ ein Schätzer für $m$ mit
$$\hat{m}_n = T_{\beta_n}\mathring{m}_n = \max\big\{\min\{\mathring{m}_n, \beta_n\}, -\beta_n\big\}$$ 
für eine Funktion
$$\mathring{m}_n(\cdot) = \mathring{m}_n(\cdot,(X_1, Y_1),\dots,(X_n, Y_n)) \in \mathcal{F}_n,$$
welche die Ungleichung
\begin{equation}
\label{lem:8:bed:1}
\frac{1}{n} \sum_{i = 1}^n |Y_i - \mathring{m}_n(X_i)|^2 \leq \frac{1}{n}\sum_{i = 1}^n |Y_i - g_{n}(X_i)|^2 + \mathrm{pen}_n(g_n)
\end{equation}
mit einer deterministischen Funktion $g_{n}\colon \R^d \to \R$ und deterministischem Penalty Term $\mathrm{pen}_n(g_{n}) \geq 0$ erfüllt.
Dann gilt für den erwarteten $L_2$-Fehler die Ungleichung
\begin{equation}
\label{lem:8:ungl}
\begin{split}
& \E \Big[\int |\hat{m}_n(x) - m(x)|^2 \mathds{P}_X(dx)\Big] \\[0.5em]
& \leq \frac{1}{n} \cdot c \cdot \log(n)^2 \cdot \Bigg(\log\bigg(\sup_{x_1^n \in (\supp(X))^n}\mathcal{N}_1\Big(\frac{1}{n \cdot \beta_n},\mathcal{F}_n,x_1^n\Big)\bigg) + 1\Bigg) \\[0.5em]
& \quad + 2 \cdot \E\bigg[\int |g_{n}(x) - m(x)|^2 \mathds{P}_X(dx) + \mathrm{pen}_n(g_{n})\bigg],
\end{split}
\end{equation}
für $n > 1$ und eine von $n$ unabhängige Konstante $c > 0$.
  \end{lem}
  \todo{Wird der Erwartungswert noch benötigt?}
Das nächste Lemma benötigen wir, um in Ungleichung~(\ref{lem:8:ungl}) die Überdeckungszahl $\mathcal{N}_1\big(\frac{1}{n \cdot \beta_n},\mathcal{F}_n,x_1^n\big)$ weiter abzuschätzen.
\begin{lem}[{\cite[Lemma 9]{kohler19}}]
\label{lem:9}
Seien $a > 0$ und $d, L, N, J_n \in \N$ so, dass $J_n \leq n^{c_{1}}$ für eine Konstante $c_1 > 0$ gilt und setze $\beta_n = c_2 \cdot \log(n)$ für eine hinreichend große Konstante $c_2 > 0.$ 
Sei die Funktion $\sigma\colon \R \to [0, 1]$ 2-zulässig. Sei $\mathcal{F}$ die Menge aller Funktionen die durch Definition~\ref{def:nn} definiert sind mit $k_1 = k_2 = \cdots = k_L = 24 \cdot (N + d)$ und einer Beschränkung des Betrags der Gewichte durch $c_{3} \cdot n^{c_{4}}$ für Konstanten $c_3, c_4 > 0$. Sei
$$ \mathcal{F}^{(J_n)} \coloneqq \biggl\{\sum_{j = 1}^{J_n} a_j \cdot f_j : f_j \in \mathcal{F} \quad \text{und} \quad \sum_{j = 1}^{J_n} a_j^2 \leq c_{5} \cdot n^{c_{6}}\biggr\}$$ für Konstanten $c_5, c_6 > 0.$
Dann gilt für $n > 1$:
$$\log\bigg(\sup_{x_1^n\in[-a,a]^{d \cdot n}} \mathcal{N}_1\Big(\frac{1}{n \cdot \beta_n}, \mathcal{F}^{(J_n)},x_1^n\Big)\bigg) \leq c \cdot \log(n) \cdot J_n,$$
für eine Konstante $c$ die nur von $L$, $N$, $a$ und $d$ abhängt.
\end{lem}

Mit den Approximationsresultaten aus diesem Abschnitt verfügen wir nun über alle Bausteine, um unser Hauptresultat zu beweisen.

\section{Der Beweis von Hauptsatz \hyperref[optstop]{3.1}}

Wir betrachten das Ereignis 
\begin{equation}
\label{event}
A_n \coloneqq \bigg[\frac{1}{n} \sum_{i = 1}^n Y_i^2 \leq 1 + \E[Y^2]\bigg].
\end{equation}

Damit können wir den erwarteten $L_2$-Fehler umschreiben zu:
\begin{equation}
\label{originaleq}
\begin{split}
\E \bigg[\int |m_n(x) - m(x)|^2 \mathds{P}_X(dx)\bigg] & = \E \bigg[\int |m_n(x) - m(x)|^2 \mathds{P}_X(dx) \cdot (\mathds{1}_{A_n^{\mathsf{c}}} + \mathds{1}_{A_n})\bigg] \\[0.5em]
& = \E \bigg[\int |m_n(x) - m(x)|^2 \mathds{P}_X(dx) \mathds{1}_{A_n^{\mathsf{c}}}\bigg] \\[0.5em]
& \quad + \E \bigg[\int |m_n(x) - m(x)|^2 \mathds{P}_X(dx) \mathds{1}_{A_n}\bigg] \\
& \eqqcolon T_{1,n} + T_{2,n}.
\end{split}
\end{equation}
In den folgenden Abschnitten kümmern wir uns um die Abschätzung der Summanden $T_{1,n}$ und $T_{2,n}$.

\subsection{Abschätzung von $T_{1,n}$}

Für zwei beliebige reelle Zahlen $u, v \in \R$ folgt aus $0 \leq (u - v)^2 = u^2 + v^2 - 2uv$ die Ungleichung $u^2 + v^2 \geq 2uv$ und damit schließlich:
\begin{equation}
\label{ungl}
\begin{split}
|(u - v)^2| & = |u^2 - 2uv + v^2|
\leq  u^2 + 2uv + v^2
\leq 2u^2 + 2v^2.
\end{split}
\end{equation}

Wir wissen, dass aufgrund der Unabhängigkeit und identischen Verteiltheit der $\R^d \times \R$-wertigen Zufallsvariablen  $(X, Y), (X_1, Y_1), (X_2, Y_2), \dots$ die Zufallsvariablen $Y_1,\dots,Y_n$ ebenso wie die Zufallsvariablen $X_1,\dots,X_n$ unabhängig und identisch verteilt sind. 
Mit Ungleichung~(\ref{ungl}) und durch die Unabhängigkeit von $A_n$ von den Zufallsvariablen $X, X_1, \dots, X_n$ erhalten wir für den ersten Summanden aus Gleichung~(\ref{originaleq}):
\begin{equation}
\label{anc}
\begin{split}
 T_{1,n} = \E \bigg[\int |m_n(x) - m(x)|^2 \mathds{P}_X(dx) \mathds{1}_{A_n^{\mathsf{c}}}\bigg] & =  \E \bigg[\int |m_n(x) - m(x)|^2 \mathds{P}_X(dx) \bigg] \cdot \mathds{P}(A_n^{\mathsf{c}}) \\[0.5em]
 & \leq \E \bigg[\int 2m_n(x)^2 + 2m(x)^2 \mathds{P}_X(dx)\bigg] \cdot \mathds{P}(A_n^{\mathsf{c}})\\[0.5em]
 & \leq \E \bigg[\int  2\beta_n^2 + 2\beta_n^2 \mathds{P}_X(dx)\bigg] \cdot \mathds{P}(A_n^{\mathsf{c}}).
\end{split}
\end{equation}
Da nach Voraussetzung $\supp(\mathds{P}_X)$ beschränkt und definitionsgemäß immer abgeschlossen ist, wissen wir nach dem Satz von Heine-Borel (siehe z.\@B.\@ \cite[Satz 5]{forster2016}), dass $\supp(\mathds{P}_X)$ kompakt ist. Da $m$ als ($p,C$)-glatte Funktion insbesondere stetig ist, wissen wir, dass sie auf einer kompakten Menge ein Maximum und Minimum annimmt. Dadurch können wir $n$ so groß wählen, dass ohne Beschränkung der Allgemeinheit $\|m\|_{\infty} \leq \beta_n$ gilt.
Da wir nach Bemerkung~\ref{mtildebeschraenkt} wissen, dass $\tilde{m}_n$ beschränkt ist, ist $m_n$ nach Konstruktion ebenfalls beschränkt. Wir haben daher bei Ungleichung~(\ref{anc}) zudem verwendet, dass $\max\{\|m\|_{\infty}, \|m_n\|_{\infty}\} \leq \beta_n$ nach Definition von $m_n$ gilt. 

Im nächsten Schritt wollen wir die Wahrscheinlichkeit~$\mathds{P}(A_n^{\mathsf{c}})$ abschätzen.
Da die Zufallsvariablen $Y_1,\dots,Y_n$ unabhängig und identisch verteilt sind, 
folgern wir daraus mit der Linearität des Erwartungswerts $\E\big[\frac{1}{n} \sum_{i = 1}^n Y_i^2\big] = \E[Y^2]$.
Mithilfe der Monotonie der Wahrscheinlichkeitsfunktion~$\mathds{P}$ und der Chebyshev-Ungleichung für $\epsilon = 1$ (siehe z.\@B.\@ \cite[Satz~5.11]{Klenke2013}) erhalten wir:
\begin{equation*}
\begin{split}
\mathds{P}(A_n^{\mathsf{c}}) = \mathds{P}\Big(\frac{1}{n}\sum_{i=1}^n Y_i^2 - \E[Y^2] \geq 1\Big)
\leq \mathds{P}\Big(\Big|\frac{1}{n}\sum_{i=1}^n Y_i^2 - \E[Y^2]\Big| \geq 1\Big)
\leq \mathds{V}\Big[\frac{1}{n}\sum_{i = 1}^nY_i^2\Big].
\end{split}
\end{equation*}
Da die Zufallsvariablen $Y_1,\dots,Y_n$ u.\@i.\@v.\@ sind, folgt mit den Rechenregeln der Varianz:
\begin{equation}
\label{tscheby}
\begin{split}
\mathds{P}(A_n^{\mathsf{c}}) \leq \frac{n \cdot  \mathds{V}[Y^2]}{n^2}
= \frac{\mathds{V}[Y^2]}{n}
= \frac{c_4}{n},
\end{split}
\end{equation}
wobei $c_4 \coloneqq \mathds{V}[Y^2]$ ist. 

Mit Ungleichung~(\ref{tscheby}) erhalten wir in Ungleichung~(\ref{anc}):
\begin{equation*}
\begin{split}
 \E \bigg[\int  2\beta_n^2 + 2\beta_n^2 \mathds{P}_X(dx)\bigg] \cdot \mathds{P}(A_n^{\mathsf{c}})
  \leq 4\beta_n^2 \cdot \mathds{P}(A_n^{\mathsf{c}}) 
  \stackrel{(\ref{tscheby})}{\leq} \frac{4 \cdot c_4 \cdot \beta_n^2}{n},
\end{split}
\end{equation*}
wobei wir bei der ersten Ungleichung verwendet haben, dass $\beta_n$ deterministisch und die Wahrscheinlichkeit~$\mathds{P}(X \in \supp(\mathds{P}_X)) = 1$ ist.
Schließlich erhalten wir in Ungleichung~(\ref{anc}), da für $n$ hinreichend groß $\log(n)^3 > 1$ gilt:
\begin{equation}
\label{6thsum}
\begin{split}
T_{1,n} \leq \frac{4 \cdot c_4 \cdot \beta_n^2}{n} & \leq c_{5} \cdot \log(n)^3 \cdot n^{-1} \leq c_{5} \cdot \log(n)^3 \cdot n^{- \frac{2p}{2p + d}},
\end{split}
\end{equation} 
mit einer von $n$ unabhängigen Konstante $c_{5} \coloneqq 4 \cdot c_4 \cdot c_1^2 > 0$.

Damit haben wir $T_{1,n}$ entsprechend der rechten Seite von Hauptsatz~\ref{optstop} abgeschätzt und wollen als Nächstes $T_{2,n}$ abschätzen.

\subsection{Abschätzung von $T_{2,n}$}

In diesem Abschnitt wollen wir Lemma~\ref{lem:8} für die Abschätzung von $T_{2,n}$ anwenden.
Sei dafür $$\hat{m}_n \coloneqq \mathds{1}_{A_n}m_n + \mathds{1}_{A_n^{\mathsf{c}}}T_{\beta_n}g_n = T_{\beta_n}\big(\mathds{1}_{A_n}\tilde{m}_n + \mathds{1}_{A_n^{\mathsf{c}}}g_n \big),$$
wobei $g_n$ definiert ist über
$$g_n(x) \coloneqq \sum_{\bi \in [M_n]^d} \sum_{\substack{ \bj \in [q]^d \\ |\bj|_1 \leq q}} \frac{1}{\bj!} \cdot \partial^{\bj}m(x_{\mathbf{i}}) \cdot f_{\net,\bj,\mathbf{i}}(x)$$
mit $x_{\bi} \in [-a_n,a_n]^d.$

Durch unsere Definition von $\hat{m}_n$ erhalten wir durch die Monotonie des Erwartungswerts und einer Abschätzung über den ganzen Raum:
\begin{equation}
\label{an}
\begin{split}
T_{2,n} = \E \bigg[\int |m_n(x) - m(x)|^2 \mathds{P}_X(dx) \mathds{1}_{A_n}\bigg] & = \E \bigg[\int |\hat{m}_n(x) - m(x)|^2 \mathds{P}_X(dx) \mathds{1}_{A_n}\bigg] \\[0.5em]
& \leq \E \bigg[\int |\hat{m}_n(x) - m(x)|^2 \mathds{P}_X(dx)\bigg].
\end{split}
\end{equation}

Da $m$ nach Voraussetzung $(p,C)$-glatt ist, existiert für alle $\alpha = (\alpha_1,\dots,\alpha_d) \in \N_0^d$ mit $\sum_{j = 1}^d\alpha_j = q$ die partielle Ableitung $\partial^{\alpha}m$. Insbesondere existiert ein $z \in \R$ mit
\begin{equation}
\label{bound}
z \coloneqq \max_{\bi \in [M_n]^d,\, \bj \in [q]^d,\, |\bj|_1 \leq q} \bigg|\frac{1}{\bj!} \cdot \partial^{\bj}m(x_{\mathbf{i}})\bigg| < \infty.
\end{equation}

Sei $\mathcal{F}$ die Menge aller Funktionen aus Definition~\ref{def:nn} mit Aktivierungsfunktion $\sigma,$ $$L = s + 2 = \lceil\log_2(N + d)\rceil + 2, \text{\quad mit } k_1 = k_2 = \cdots = k_L = 24 \cdot (N + d)$$ und einer Konstante~$c_6 > 0$ so, dass der Betrag der Gewichte durch $n^{c_{6}}$ beschränkt ist. Wir definieren
$$ \mathcal{F}^{(J_n)} \coloneqq \biggl\{\sum_{j = 1}^{J_n} a_j \cdot f_j : f_j \in \mathcal{F} \text{ und } \sum_{j = 1}^{J_n} a_j^2 \leq c_7 \cdot n \biggr\}$$
mit 
$$
J_n = (M_n + 1)^d \cdot \bigg|\Bigl\{\bj \in [N]^d : |\bj|_1 \leq N \Bigr\}\bigg|$$
und
\begin{equation}
\label{constantc21}
\begin{split}
c_7 \coloneqq \max\Biggl\{\frac{1 + \E[Y^2]}{c_8}, c_2^d \cdot (N + 1)^d \cdot z^2\Biggr\}.
\end{split}
\end{equation}
Hierbei bezeichnet $c_8$ die Konstante des Regularitätsterms aus Gleichung~(\ref{min}). Da wir uns in der nichtparametrischen Regressionsschätzung befinden, gilt unter anderem die Bedingung $\E[Y^2] < \infty$ und daher ist $c_7$ auch wohldefiniert. 
Weiterhin folgt wie in Kapitel~\ref{subsec:2.2} mit $S = J_n$,
\begin{equation}
\label{jn}
J_n =  (M_n + 1)^d \cdot \binom{N + d}{d} \leq (M_n + 1)^d \cdot (N + 1)^d.
\end{equation}

Um Lemma~\ref{lem:8} anwenden zu können, zeigen wir im Folgenden zunächst $g_n, \tilde{m}_n \in \mathcal{F}^{(J_n)}.$

Da nach Konstruktion $f_{\net,\bj,\mathbf{i}} \in \mathcal{F}$ ist, folgt mit Gleichung~(\ref{constantc21}), dass für $n$ hinreichend groß die Abschätzung
\begin{equation*}
\begin{split}
\sum_{\bi \in [M_n]^d} \sum_{\substack{ \bj \in [q]^d \\ |\bj|_1  \leq q}} \big| \frac{1}{\bj!} \cdot \partial^{\bj}m(x_{\mathbf{i}})\big|^2 
& \leq (M_n + 1)^d (N + 1)^d \cdot z^2 \\
& \leq (2 c_2 \cdot n^{1/2p + d})^d \cdot (N + 1)^d \cdot z^2 \\
& \leq c_7 \cdot n
\end{split}
\end{equation*}
gilt und damit $g_n$ in $\mathcal{F}^{(J_n)}$ liegt. 

Wir zeigen nun $\tilde{m}_n \in \mathcal{F}^{(J_n)}$.
Nach Gleichung~(\ref{umschreiben}) können wir unseren Schätzer $\tilde{m}_n$ darstellen durch:
$$\tilde{m}_n(x) = \sum_{j = 1}^{J_n}\hat{a}_j \cdot f_j$$
für geeignete $f_j \in \mathcal{F}$ und $\hat{a}_j$, welche die Ungleichung
\begin{equation*}
\begin{split}
\frac{c_8}{n}\sum_{j = 1}^{J_n} \hat{a}_j^2 & = \frac{c_8}{n} \sum_{\bi \in [M_n]^d} \sum_{\substack{\bj \in [q]^d \\ |\bj|_1 \leq q}} a_{\mathbf{i},\bj}^2
\leq \frac{1}{n} \sum_{i = 1}^n|Y_i - \tilde{m}_n(X_i)|^2 + \frac{c_8}{n} \sum_{\bi \in [M_n]^d} \sum_{\substack{\bj \in [q]^d \\ |\bj|_1 \leq q}} a_{\mathbf{i},\bj}^2 
\leq \sum_{i = 1}^n Y_i^2
\end{split}
\end{equation*}
erfüllen, wobei wir bei der letzten Ungleichung wie in Kapitel~\ref{chap:2} die minimierende Eigenschaft von $a_{\mathbf{i},\bj}$ verwendet haben und zum Schluss die Koeffizienten Null gesetzt haben. Da $c_8 > 0$ ist, erhalten wir damit, dass die Koeffizienten $\hat{a}_j$ die Eigenschaft
\begin{equation}
\label{eq:foran}
\sum_{j = 1}^{J_n} \hat{a}_j^2  \leq \frac{1}{n}\sum_{i = 1}^n Y_i^2 \cdot \frac{n}{c_8}
\end{equation}
erfüllen müssen.
Da Ungleichung~\eqref{eq:foran} insbesondere auf dem Ereignis $A_n$ gilt, folgt aus der Definition von $A_n$ in Gleichung~(\ref{event}) und der Definition der Konstante $c_7$ in Gleichung~(\ref{constantc21}):
$$\sum_{j = 1}^{J_n}\hat{a}_j^2 \leq \frac{1 + \E[Y^2]}{c_8} \cdot n \leq c_7 \cdot n,$$
woraus durch $f_j \in \mathcal{F}$ schließlich $\tilde{m}_n \in \mathcal{F}^{(J_n)}$ folgt.

Als Nächstes wollen wir Ungleichung~(\ref{lem:8:bed:1}) aus Lemma~\ref{lem:8} für $\tilde{m}_n$ und $g_n$ zeigen. Die Funktionen $\tilde{m}_n$ und $g_n$ unterscheiden sich in den Vorfaktoren von $f_j$. Die Koeffizienten $a_{\mathbf{i},\bj}$ von $\tilde{m}_n$ haben wir durch Minimierung des Funktionals $\varphi$ aus Lemma~\ref{mincoef} erhalten. Nach Voraussetzung ist $N \geq q$ und damit gilt dann insbesondere $\{0,\dots,q\} \subseteq \{0,\dots,N\}$, daher wurden bei der Minimierung des Funktionals unter anderem die Koeffizienten von $g_n$ betrachtet. Daher erhalten wir:
\begin{equation}
\begin{split}
\label{tilde}
\frac{1}{n} \sum_{i = 1}^n|Y_i - \tilde{m}_n(X_i)|^2 
& \leq \frac{1}{n} \sum_{i = 1}^n|Y_i - \tilde{m}_n(X_i)|^2 + \frac{c_8}{n} \cdot \sum_{\bi \in [M_n]^d} \sum_{\substack{ \bj \in [N]^d \\ |\bj|_1 \leq N}} a_{\mathbf{i},\bj}^2 \\[0.5em]
& \leq \frac{1}{n} \sum_{i = 1}^n|Y_i - g_n(X_i)|^2 + \frac{c_8}{n} \cdot \sum_{\bi \in [M_n]^d} \sum_{\substack{ \bj \in [q]^d \\ |\bj|_1 \leq q}} \bigg|\frac{1}{\bj!} \cdot \partial^{\bj}m(x_{\mathbf{i}})\bigg|^2 \\[0.5em]
& \leq \frac{1}{n} \sum_{i = 1}^n|Y_i - g_n(X_i)|^2 + c_{9} \cdot \frac{(M_n + 1)^d}{n},
\end{split}
\end{equation}
mit $c_{9} = c_8 \cdot (q + 1)^d \cdot z^2$ als Konstante, die unabhängig von $n$ ist.
Wir erhalten damit für $\xoverline{m}_n \in \{\tilde{m}_n, g_n\} \subseteq \mathcal{F}^{(J_n)}$:
\begin{equation}
\label{bed}
\frac{1}{n} \sum_{i = 1}^n|Y_i - \xoverline{m}_n(X_i)|^2 \leq \frac{1}{n} \sum_{i = 1}^n|Y_i - g_n(X_i)|^2 + c_9 \cdot \frac{(M_n + 1)^d}{n},
\end{equation}
da für $\xoverline{m}_n = g_n$ die Ungleichung unmittelbar folgt.
Da $g_n$ nach Definition deterministisch, damit also unabhängig von $(X_1, Y_1),\dots,(X_n, Y_n)$ ist, sind mit der Abschätzung~(\ref{bed}) für $\hat{m}_n = T_{\beta_n}\xoverline{m}_n$ mit $\xoverline{m}_n \in \mathcal{F}^{(J_n)}$ und dem \emph{Penalty Term} $\mathrm{pen}_n(g_{n}) = c_9 \cdot \frac{(M_n + 1)^d}{n} > 0$ die Voraussetzungen für Lemma~\ref{lem:8} erfüllt. Wir erhalten durch dessen Anwendung:
\begin{equation}
\label{eq:lem8}
\begin{split}
& \E \Big[\int |\hat{m}_n(x) - m(x)|^2 \mathds{P}_X(dx)\Big] \\
& \leq \frac{1}{n} \cdot c_9 \cdot \log(n)^2 \cdot \Bigg(\log\bigg(\sup_{x_1^n \in (\supp(X))^n}\mathcal{N}_1\Big(\frac{1}{n \cdot \beta_n},\mathcal{F}^{(J_n)},x_1^n\Big)\bigg) + 1\Bigg) \\[0.5em]
& \quad + 2 \cdot \E\bigg[\int |g_{n}(x) - m(x)|^2 \mathds{P}_X(dx) + c_9 \cdot \frac{(M_n + 1)^d}{n}\bigg] \\[0.5em]
& = \frac{1}{n} \cdot c_9 \cdot \log(n)^2 \cdot \Bigg(\log\bigg(\sup_{x_1^n \in (\supp(X))^n}\mathcal{N}_1\Big(\frac{1}{n \cdot \beta_n},\mathcal{F}^{(J_n)},x_1^n\Big)\bigg) + 1\Bigg) \\[0.5em]
& \quad + 2 \int |g_{n}(x) - m(x)|^2 \mathds{P}_X(dx) + 2 \cdot c_9 \cdot \frac{(M_n + 1)^d}{n} \\
& \eqqcolon T_{2,A,n} + 2 \cdot T_{2,B,n} + 2 \cdot c_9 \cdot \frac{(M_n + 1)^d}{n},
\end{split}
\end{equation}
wobei wir bei der letzten Gleichheit verwendet haben, dass der letzte Summand deterministisch ist. Zudem wissen wir, dass $c_9$ unabhängig von $n$ ist und $n > 1$, da wir $n$ hinreichend groß wählen.
Des Weiteren erhalten wir, da für $n$ hinreichend groß $\log(n)^3 > 1$ gilt:
\begin{equation}
\label{5thsum}
\begin{split}
c_9 \cdot \frac{(M_n + 1)^d}{n} & \leq c_{10} \cdot \frac{n^{\frac{d}{2p + d}}}{n} \leq c_{10} \cdot \log(n)^3 \cdot n^{- \frac{2p}{2p + d}} ,
\end{split}
\end{equation} 
mit einer von $n$ unabhängigen Konstanten $c_{10} \coloneqq c_9 \cdot (2c_2)^d > 0$.
Mithilfe von Ungleichung~(\ref{5thsum}) erhalten wir nun in Gleichung~(\ref{eq:lem8}):
$$
\E \Big[\int |\hat{m}_n(x) - m(x)|^2 \mathds{P}_X(dx)\Big] \leq T_{2,A,n} + 2 \cdot T_{2,B,n} + 2 \cdot c_{10} \cdot \log(n)^3 \cdot n^{- \frac{2p}{2p + d}}.
$$

Wir wollen im Folgenden die Summanden $T_{2,A,n}$ und $T_{2,B,n}$ weiter abschätzen.

\subsubsection{Abschätzung von $T_{2,A,n}$}
Als Erstes überprüfen wir die Voraussetzungen von Lemma~\ref{lem:9}, um damit dann den Summanden, welcher die Überdeckungszahl $\mathcal{N}_1$ enthält, in Ungleichung~(\ref{eq:lem8}) weiter abzuschätzen.
Nach Voraussetzung ist $\beta_n = c_1 \cdot \log(n)$ und $a_n = (\log n)^{1/(6(N + d))} > 0$ für hinreichend großes $n$. Nach Voraussetzung sind zudem $d, N, J_n \in \N$ und es gilt nach Gleichung~(\ref{jn}): 
$$J_n \leq (M_n + 1)^d \cdot (N + 1)^d \leq n^{\gamma},$$
für hinreichend großes $n$ und Konstante $\gamma > 0$. Wir betrachten hier den logistischen Squasher $\sigma$ welcher nach Lemma~\ref{lem:logsquasher} insbesondere 2-zulässig ist. Da die hier betrachtete Menge von Funktionen $\mathcal{F}^{(J_n)}$ identisch mit der aus Lemma~\ref{lem:9} ist, sind nun alle Voraussetzungen für Lemma~\ref{lem:9} erfüllt. Nach Voraussetzung wissen wir, dass $\supp(\mathds{P}_X)$ beschränkt ist und wir können $n$ so groß wählen, dass wir ohne Beschränkung der Allgemeinheit annehmen können, dass $\supp(\mathds{P}_X) = \{x \in \R^d \mid \forall \epsilon > 0 : \mathds{P}_X(S_{\epsilon}(x)) > 0\} \subseteq [-a_n, a_n]^d$ ist, mit $S_{\epsilon}$ als $\epsilon$-Umgebung um $x \in \R^d$. In der nächsten Ungleichungskette bezeichnen wir mit $c$ die Konstante aus Lemma~\ref{lem:9}. Wir erhalten damit durch Lemma~\ref{lem:9} für hinreichend großes $n$:
\begin{equation}
\label{lem9sol}
\begin{split}
T_{2,A,n} = & \frac{1}{n} \cdot c_9 \cdot \log(n)^2 \cdot \Bigg(\log\bigg(\sup_{x_1^n \in (\supp(X))^n}\mathcal{N}_1\Big(\frac{1}{n \cdot \beta_n},\mathcal{F}^{(J_n)},x_1^n\Big)\bigg) + 1\Bigg) \\[0.5em]
& \leq \frac{1}{n} \cdot c_9 \cdot \log(n)^2 \cdot \Big(\big(c \cdot \log(n) \cdot J_n \big) + 1\Big) \\
& \leq \frac{1}{n} \cdot c_9 \cdot \log(n)^2 \cdot \big(2 \cdot c \cdot \log(n) \cdot J_n\big) \\
& \leq c_{11} \cdot \frac{1}{n} \cdot \log(n)^3 \cdot J_n,
\end{split}
\end{equation}
wobei $c_{11} \coloneqq 2 c_9 \cdot c$ eine von $n$ unabhängige Konstante ist.
Durch Einsetzen der Definition von $M_n$ erhalten wir für hinreichend großes $n$ 
$$
(M_n + 1)^d \leq (c_2 \cdot n^{\frac{1}{2p + d}} + 2)^d \leq 2^d \cdot c_2^d \cdot n^{\frac{d}{2p + d}}
$$
und damit schließlich für Ungleichung~(\ref{lem9sol}):
\begin{equation}
\begin{split}
c_{11} \cdot \frac{1}{n} \cdot \log(n)^3 \cdot J_n
& \leq c_{11} \cdot (N+1)^d \cdot 2^d \cdot c_2^d \cdot \log(n)^3 \cdot n^{-\frac{2p}{2p + d}} \\
& = c_{12} \cdot \log(n)^3 \cdot n^{-\frac{2p}{2p + d}},
\end{split}
\end{equation}
mit einer von $n$ unabhängigen Konstante $c_{12} \coloneqq c_{11} \cdot (N + 1)^d \cdot 2^d \cdot c_2^d > 0.$
Damit haben wir den Summanden aus Ungleichung~(\ref{eq:lem8}), welcher die Überdeckungszahl $\mathcal{N}_1$ enthält, abschließend passend zum Endresultat abgeschätzt.
%Als Nächstes wollen wir den Term
%$$
%\int |g_{n}(x) - m(x)|^2 \mathds{P}_X(dx),
%$$
%aus Ungleichung~(\ref{eq:lem8}), weiter abschätzen.

\subsubsection{Abschätzung von $T_{2,B,n}$}
Sei 
\begin{equation*}
\begin{split}
P_{m,n}(x) \coloneqq \sum_{\bi \in [M_n]^d} \sum_{\substack{ \bj \in [q]^d \\ |\bj|_1 \leq q}} \frac{1}{\bj!} \cdot \partial^{\bj} m (x_{\mathbf{i}}) \cdot (x - x_{\mathbf{i}})^{\bj} \prod_{j = 1}^d\Big(1 - \frac{M_n}{2a_n} \cdot \Big|x^{(j)} - x_{\mathbf{i}}^{(j)}\Big|\Big)_+
\end{split}
\end{equation*}
wie in Gleichung~(\ref{konvexkomb}) eine lokale Spline-Interpolation von Taylorpolynomen von $m$.
Mit Ungleichung~(\ref{ungl}) zusammen mit einer Nulladdition und der Linearität des Integrals erhalten wir:
\begin{equation*}
\begin{split}
T_{2,B,n} = & \int |g_n(x) - m(x)|^2 \mathds{P}_X(dx) \\
& = \int |g_n(x) - P_{m,n}(x) + P_{m,n}(x) - m(x)|^2 \mathds{P}_X(dx) \\
& \leq \int 2 |g_n(x) - P_{m,n}(x)|^2 \mathds{P}_X(dx) + \int 2 |P_{m,n}(x) - m(x)|^2 \mathds{P}_X(dx).
\end{split}
\end{equation*}
Aus der Supremumseigenschaft folgt weiter
\begin{equation}
\label{gnmx}
\begin{split}
&  \int \sup_{x \in [-a_n, a_n]^d} |g_n(x) - P_{m,n}(x)|^2 \mathds{P}_X(dx) +  \int \sup_{x \in [-a_n, a_n]^d} |P_{m,n}(x) - m(x)|^2 \mathds{P}_X(dx) \\[0.5em]
& =  \sup_{x \in [-a_n, a_n]^d} |g_n(x) - P_{m,n}(x)|^2 +  \sup_{x \in [-a_n, a_n]^d} |P_{m,n}(x) - m(x)|^2  \\[0.5em]
& \eqqcolon  \sup_{x \in [-a_n, a_n]^d} T_{2,B_1,n}^2(x) +  \sup_{x \in [-a_n, a_n]^d} T_{2,B_2,n}^2(x),
\end{split}
\end{equation}
wobei wir im ersten Schritt $\supp(\mathds{P}_X) \subseteq [-a_n, a_n]^d$ und $\mathds{P}(X \in \supp(\mathds{P}_X)) = 1$ verwendet haben. Damit folgt schließlich:
$$T_{2,B,n} \leq 2 \cdot \Big(\sup_{x \in [-a_n, a_n]^d} T_{2,B_1,n}^2(x) +  \sup_{x \in [-a_n, a_n]^d} T_{2,B_2,n}^2(x)\Big).$$
Um die letzten beiden Summanden der Gleichung~(\ref{gnmx}) weiter abzuschätzen, möchten wir Lemma~\ref{lem:5} anwenden. 

\paragraph*{Abschätzung von $T_{2,B_1,n}$}\mbox{}\\
Wir überprüfen, ob für Lemma~\ref{lem:5} alle Voraussetzungen erfüllt sind. Wir betrachten wieder den logistischen Squasher $\sigma$ aus Gleichung~\ref{logsquasher}, welcher nach Lemma~\ref{lem:logsquasher} insbesondere 2-zulässig ist. Zudem ist für hinreichend großes $n$ die Bedingung 
\begin{equation*}
\begin{split}
R_n & \geq \max\biggl\{\frac{\|\sigma''\|_{\infty} \cdot (M_n + 1)}{2 \cdot |\sigma'(t_{\sigma})|}, \frac{9 \cdot \|\sigma''\|_{\infty} \cdot a_n}{|\sigma'(t_{\sigma})|}, \\[0.5em]
& \quad \frac{20 \cdot \|\sigma'''\|_{\infty}}{3 \cdot |\sigma''(t_{\sigma})|} \cdot 3^{3 \cdot 3^s} \cdot a_n^{3 \cdot 2^s}, 1792 \cdot \frac{\max\{\|\sigma''\|_{\infty},\|\sigma'''\|_{\infty}, 1\}}{\min\{2 \cdot |\sigma'(t_{\sigma})|, |\sigma''(t_{\sigma})|, 1\}} \cdot M_n^3 \biggr\}
\end{split}
\end{equation*}
\todo{passt das mit Lemma 2.8 zusammen?}
erfüllt. Daher sind für unser neuronales Netz aus Definition \ref{fnet} mit $x_{\mathbf{i}} \in [-a_n, a_n]^d$ alle Voraussetzungen für Lemma~\ref{lem:5} erfüllt.
Wir erhalten damit für $x \in [-a_n ,a_n]^d$ und $n$ hinreichend groß:
\begin{equation}
\label{vorbereitung}
\begin{split}
\bigg| f_{\net,\bj,\mathbf{i}}(x) - (x - x_{\mathbf{i}})^{\bj} \cdot \prod_{j = 1}^d(1 - \frac{M_n}{2a_n} \cdot |x^{(j)} - x_{\mathbf{i}}^{(j)}|)_+ \bigg| & \leq c \cdot 3^{3 \cdot 3^s} \cdot a_n^{3 \cdot 2^s} \cdot M_n^3 \cdot \frac{1}{R_n} \\[0.5em]
& \leq c\cdot a_n^{3 \cdot (N + d) \cdot 2} \cdot \frac{M_n^3}{R_n} \\[0.5em]
& = c\cdot \log(n) \cdot \frac{M_n^3}{R_n},  
\end{split}
\end{equation}
wobei wir in $c$ alle von $n$ unabhängigen Konstanten zusammenfassen. Diese Konstante enthält unter anderem die Konstante aus Lemma~\ref{lem:5}.
Des Weiteren haben wir verwendet, dass für hinreichend großes $n$ die Ungleichung
$$a_n^{2^{\lceil\log_2(N + d)\rceil}} \leq a_n^{2^{\log_2(N + d) + 1}} = a_n^{(N + d) \cdot 2}$$ gilt. Im letzten Schritt haben wir in Ungleichung~(\ref{vorbereitung}) die Definition von $a_n$ eingesetzt. 
Mit Ungleichung~(\ref{vorbereitung}) und Gleichung~(\ref{bound}) erhalten wir nun:
\begin{equation}
\label{1stsum}
\begin{split}
T_{2,B_1,n}(x) & = |g_n(x) - P_{m,n}(x)| \\
& = \bigg|\! \sum_{\bi \in [M_n]^d} \sum_{\substack{ \bj \in [q]^d \\ |\bj|_1 \leq q}} \frac{1}{\bj!} \cdot \partial^{\bj} m (x_{\mathbf{i}}) \bigg|\!\! \cdot \!\!\bigg| f_{\net,\bj,\mathbf{i}}(x) - (x - x_{\mathbf{i}})^{\bj} \cdot \prod_{j = 1}^d\Big(1 - \frac{M_n}{2a_n} \cdot \Big|x^{(j)} - x_{\mathbf{i}}^{(j)}\Big|\Big)_+ \bigg| \\[0.5em]
& \leq (M_n + 1)^d \cdot (q + 1)^d \cdot z \cdot \bigg| f_{\net,\bj,\mathbf{i}}(x) - (x - x_{\mathbf{i}})^{\bj} \cdot \prod_{j = 1}^d\Big(1 - \frac{M_n}{2a_n} \cdot \Big|x^{(j)} - x_{\mathbf{i}}^{(j)}\Big|\Big)_+ \bigg| \\[0.5em]
& \leq  (M_n + 1)^d \cdot c\cdot \log(n) \cdot \frac{M_n^3}{R_n},
\end{split}
\end{equation}
wobei wir in $c$ alle von $n$ unabhängigen Konstanten zusammenfassen, was insbesondere die Konstante $c$ aus Lemma~\ref{lem:5} einschließt.
Durch Einsetzen der Definitionen von $M_n$ und $R_n$ erhalten wir für $n$ hinreichend groß:
$$
\frac{(M_n + 1)^{8d}}{R_n^2} \leq c \cdot \frac{n^{\frac{8d}{2p + d}}}{n^{2d + 8}} = c \cdot n^{\frac{8d}{2p + d} - 2d -8} \leq c \cdot n^{\frac{8d}{2p + d}  -8\frac{2p + d}{2p + d}} = c \cdot n^{-\frac{16p}{2p + d}} \leq c \cdot n^{-\frac{2p}{2p + d}},
$$
wobei in dieser Ungleichungskette $c = (2c_2)^{8d}$ gilt.
Bei der letzten Ungleichung haben wir verwendet, dass für $p > 0$ auch $\frac{16p}{2p + d} > \frac{2p}{2p + d}$ gilt. Damit erhalten wir für alle $x \in [-a_n,a_n]^d$:
\begin{equation}
\label{3rdsum}
\begin{split}
|g_n(x) - P_{m,n}(x)|^2 & \leq \bigg((M_n + 1)^d \cdot c \cdot \log(n) \cdot \frac{M_n^3}{R_n}\bigg)^2 \\[0.5em]
& \leq c^2 \cdot (M_n + 1)^{2d} \cdot \log(n)^2 \cdot \frac{M_n^6}{R_n^2} \\[0.5em]
& \leq  c^2 \cdot (M_n + 1)^{2d} \cdot \log(n)^2 \cdot \frac{(M_n + 1)^{6d}}{R_n^2} \\[0.5em]
& \leq  c^2 \cdot \log(n)^2 \cdot \frac{(M_n + 1)^{8d}}{R_n^2} \\[0.5em]
& \leq c_{13} \cdot n^{-\frac{2p}{2p + d}} \cdot \log(n)^3,
\end{split}
\end{equation}
mit $c_{13} = c^2 \cdot (2c_2)^{8d} > 0$ und unabhängig von $n$ ist. Bei der letzten Ungleichung haben wir verwendet, dass $\log(n)^2 < \log(n)^3$ für $n$ hinreichend groß gilt. Wenden wir schließlich noch das Supremum über $x \in [-a_n, a_n]^d$ auf Ungleichung\eqref{3rdsum} an, erhalten wir wie verlangt eine Abschätzung für $\sup_{x \in [-a_n, a_n]^d} T_{2,B_1,n}(x)^2$.

\paragraph*{Abschätzung von $T_{2,B_2,n}$}\mbox{}\\
Da nach Konstruktion $a_n > 0$, $m$ $(p, C)$-glatt und $P_{m,n}(x)$ nach Lemma~\ref{lem:loccon} eine Spline"=Interpolation von Taylorpolynomen von $m$ ist, erhalten wir mit Lemma~\ref{lem:pcsmooth} für $x \in [-a_n,a_n]^d$:
\begin{equation}
\label{2ndsum}
\begin{split}
T_{2,B_2,n}(x) = |P_{m,n}(x) - m(x)| \leq c_{14} \cdot \frac{a_n^p}{M_n^p} \leq c_{14} \cdot \log(n) \cdot \frac{1}{M_n^p}.
\end{split}
\end{equation}
In dieser Ungleichung ist $c_{14} > 0$ eine von $n$ unabhängige Konstante aus Lemma~\ref{lem:pcsmooth} und wir haben zudem verwendet, dass:
$$a_n^p = a_n^{q + s} \leq a_n^{N + d} \leq a_n^{6 \cdot (N + d)} = \log(n),$$
für $p = q + s$ für hinreichend großes $n$ gilt, da nach Voraussetzung $N \geq q$ und $d \geq s$ mit $s \in (0, 1]$ ist.
Durch Quadrieren bleibt die Ungleichung~(\ref{2ndsum}) auch erhalten und, da in der Ungleichung die rechte Seite unabhängig von $x$ ist, gilt die Ungleichung ebenfalls für das Supremum über $x \in [-a_n,a_n]^d$. Für $n$ hinreichend groß, erhalten wir durch Einsetzen der Definition von $M_n$:
\begin{equation}
\label{4thsum}
\begin{split}
\sup_{x \in [-a_n, a_n]^d} |P_{m,n}(x) - m(x)|^2 & \leq \bigg(c_{14} \cdot \log(n) \cdot \frac{1}{M_n^p}\bigg)^2 \\
& \leq c_{14}^2 \cdot \log(n)^2 \cdot c_{2}^{-2p} \cdot n^{-\frac{2p}{2p + d}} \\[0.5em]
& \leq \Big(\frac{c_{14}}{c_2^p}\Big)^2 \cdot \log(n)^3 \cdot n^{-\frac{2p}{2p + d}}.
\end{split}
\end{equation}
Hiermit haben wir $\sup_{x \in [-a_n, a_n]^d} T_{2,B_2,n}(x)^2$ aus Ungleichung~(\ref{gnmx}) abgeschätzt.
Dadurch wurde schließlich auch $T_{2,B,n}$ und damit auch $T_{2,n}$ abgeschätzt.  

Mit der Abschätzung von $T_{1,n}$ haben wir nun alle Summanden von Ungleichung ($\ref{originaleq}$) abgeschätzt und erhalten schließlich:
\begin{equation*}
\E \Big[\int |m_n(x) - m(x)|^2 \mathds{P}_X(dx)\Big] \leq c_{\mathrm{fin}} \cdot \log(n)^3 \cdot n^{- \frac{2p}{2p + d}},
\end{equation*}
mit 
$$c_{\mathrm{fin}} = c_{5} + 2 \cdot c_{10} + c_{12} + 2 \cdot \Bigg(2 \cdot \bigg( \Big(\frac{c_{14}}{c_2^p}\Big)^2 +  c_{13}\bigg)\Bigg),$$
wobei $c_{\mathrm{fin}}$ als Summe nichtnegativer und positiver Konstanten, die unabhängig von $n$ sind, nichtnegativ und unabhängig von $n$ ist.
Damit haben wir unser Hauptresultat bewiesen. $\hfill\square$ 	