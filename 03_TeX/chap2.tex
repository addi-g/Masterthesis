\chapter{Konstruktion des Neuronale Netze Regressionsschätzers}
\label{chap:2}

In diesem Kapitel werden wir mithilfe von unseren gegebenen Datenmenge 
$$ \mathfrak{D}_n = \{(X_1, Y_1),\dots,(X_n, Y_n)\},$$
unseren Regressionsschätzer konstruieren. 

Die Netzwerkarchitektur $(L, k)$ hängt von einer positiven ganzen Zahl $L$, die der Anzahl der verborgenen Schichten ist und einem Vektor $k = (k_1,\dots,k_L) \in \N^L$, der mit jeder Kombonente die Anzahl der Neuronen in der jeweiligen verborgen Schichte angibt. 

Ein mehrschichtiges feedforward neuronales Netz mit Architektur $(L, k)$ und dem logistischen squasher (\ref{logsquasher}) als Aktivierungsfunktion, ist eine reelwertige Funktion $f\colon \R^d \to \R$ definiert durch$\colon$

\begin{align}
\label{networkarch}
f(x) & = \sum_{i = 1}^{k_L} c_i^{(L)} \cdot f_i^{(L)}(x) + c_0^{(L)} \nonumber \\
\intertext{für $c_0^{(L)},\dots,c_{k_L}^{(L)} \in \R$ und für $f_i^{(L)}$ rekursiv definiert durch$\colon$} \nonumber \\
f_i^{(r)}(x) & = \sigma \bigg(\sum_{j = 1}^{k_r - 1} c_{i,j}^{(r - 1)} \cdot f_j^{(r - 1)}(x) + c_{i,0}^{(r - 1)} \bigg) \nonumber \\
\intertext{für $c_{i,0}^{(r - 1)},\dots,c_{i,k_{r - 1}}^{(r - 1)} \in \R (r = 2,\dots, L)$ und$\colon$} \\
f_i^{(1)}(x) & = \sigma \Bigg(\sum_{j = 1}^{d} c_{i,j}^{(0)} \cdot x^{(j)} + c_{i,0}^{(0)} \Bigg) \nonumber \\
\intertext{für $c_{i,0}^{(0)},\dots,c_{i,d}^{(0)} \in \R$.} \nonumber
\end{align} 
Bei neuronale Netze Regressionsschätzer wählt man keine Aktivierungsfunktion mehr, da wir einen Funktionswert schätzen wollen nichts mit einer Wahrscheinlichkeit klassifizieren möchten. (REFERENZ)

Für die Konstruktion unseren Schätzers verwenden wir die gegebene Datenmenge $\mathfrak{D}$ und wählen die Gewichte des neuronalen Netzes so, dass die resultierende Funktion aus (\ref{networkarch}) eine gute Schätzungen für die Regressionsfunktion ist. Dafür wählen wir die Gewichte bis auf die in der Ausgabeschicht fest und schätzen die Gewichte in der Ausgabeschicht in dem wir mit unserer Datenmenge ein regularisiertes Kleinste-Quadrate-Problem (REFERENZ) lösen.

\section{Definition der Netzwerkarchitektur}
Sei $a > 0$ fest. Die Wahl der Netzwerkarchitektur und der Werte aller Gewichte bis auf die aus der Ausgabeschicht ist durch folgendes Approximationsresultat durch eine lokale Konvexkombination von Taylorpolynomen für $(p,C)$-glatte Funktionen für $x \in [-a, a]^d$ motiviert. 
Sei dafür $M \in \N$ und $\mathbf{i} = (i^{(1)},\dots,i^{(d)}) \in \{0,\dots, M\}^d$, sei
$$x_{\mathbf{i}} = \bigg( -a + i^{(1)} \cdot \frac{2a}{M},\dots, -a + i^{(d)} \cdot \frac{2a}{M}\bigg)$$
und sei
$$\{\mathbf{i}_1,\dots, \mathbf{i}_{(M + 1)^d}\} = \{0,\dots,M\}^d,$$
d.h.  $\mathbf{i}_1,\dots,\mathbf{
i}_{(M + 1)^d}$ sind insgesamt $M + 1$ Vektoren der Dimension $d$, wobei jede Komponente aus der Menge $\{0,\dots,M\}$ ausgewählt wurde.
Für $k \in \{1,\dots,(M + 1)^d\}$ sei
$$p_{\mathbf{i}_k}(x) = \sum_{\substack{j_1,\dots,j_d \in \{0,\dots,q\} \\j_1+\dots +j_d \leq q}} \frac{1}{j_1! \cdots j_d!} \cdot \frac{\partial^{j_1+\cdots + j_d} m}{\partial^{j_1} x^{(1)}\cdots \partial^{j_d} x^{(d)}}(x_{\mathbf{i}_k}) \cdot (x^{(1)} - x_{\mathbf{i}_k}^{(1)})^{j_1} \cdots (x^{(d)} - x_{\mathbf{i}_k}^{(d)})^{j_d}$$
das Taylorpolynom von $m$ der Ordnung $q$ im Entwicklungspunkt $x_{\mathbf{i}_k}$ und sei
\begin{equation}
\label{konvexkomb}
P(x) = \sum_{k = 1}^{(M + 1)^d} p_{\mathbf{i}_k}(x) \prod_{j = 1}^{d} \bigg(1 - \frac{M}{2a} \cdot |x^{(j)} - x_{\mathbf{i}_k}^{(j)}|\bigg)_+,
\end{equation}
 mit $z_+ = \max\{z, 0\} (z \in \R).$
Wir zeigen im folgenden Lemma dass $P(x)$ eine lokale Konvexkombination von Taylorpolynomen von m ist.
\begin{lem}
\label{lem:loccon}
Sei $a >0, M \in \N$ und $$P(x) = \sum_{k = 1}^{(M + 1)^d} p_{\mathbf{i}_k}(x) \prod_{j = 1}^{d} \bigg(1 - \frac{M}{2a} \cdot |x^{(j)} - x_{\mathbf{i}_k}^{(j)}|\bigg)_+$$ mit $p_{\mathbf{i}_k}(x)$ wie oben, dann ist $P(x)$ eine lokale Konvexkombination von Taylorpolynomen von m.
\end{lem}
\begin{proof}
Es sind drei Bedingungen zu überprüfen. Als erstes geben wir aber für $d = 2$ und $M = 3$ eine Skizze an um die Idee des Beweises zu veranschaulichen. (SKIZZE EINFÜGEN)
Es ist ein Gitter mit $(M + 1)^d$ Gitterpunkten die den $x_{\mathbf{i}_k}$ entsprechen. Der Abstand zwischen zwei Gitterpunkten beträgt $\frac{2a}{M}.$ Man betrachtet immer den Abstand zu den nähesten $2^d$ Gitterpunkten, da  $(1 - \frac{M}{2a} \cdot |x^{(j)} - x_{\mathbf{i}_k}^{(j)}|)_+ = 0$ immer dann gilt, wenn der Abstand zwischen $x^{(j)}$ und $x_{\mathbf{i}_k}^{(j)}$ größer als $\frac{2a}{M}$ ist.     

\textbf{i)} Im folgenden wollen wir $$\sum_{k = 1}^{(M + 1)^d} \prod_{j = 1}^d \bigg(1 - \frac{M}{2a} \cdot |x^{(j)} - x_{\mathbf{i}_k}^{(j)}|\bigg)_+ = 1,$$ per Induktion über $d$ zeigen. 
	
	 \emph{Induktionsanfang} (IA): Für $d = 1$ kann nur zwischen zwei Gitterpunkten liegen und mit der obigen Begründung ist der Rest gleich Null, daher nehmen wir oBdA an, dass $x$ zwischen $x_{\mathbf{i}_1}$ und $x_{\mathbf{i}_2}$ liegt. 
	Damit folgt 
	\begin{equation*}
	\begin{split}
	\sum_{k = 1}^{(M + 1)} (1 - \frac{M}{2a} \cdot |x - x_{\mathbf{i}_k}|)_+ & = (1 - \frac{M}{2a} \cdot |x - x_{\mathbf{i}_1}|)_+ + (1 - \frac{M}{2a} \cdot |x - x_{\mathbf{i}_2}|)_+ \\
	& = 1 + 1 - \frac{M}{2a} \cdot |x - x_{\mathbf{i}_1} + x_{\mathbf{i}_2} - x| \\
	& = 1 + 1 - \frac{M}{2a} \cdot \frac{2a}{M} \\
	& = 1,
	\end{split}
	\end{equation*} wobei wir unter anderem verwendet haben, dass beide Summenden unabhängig von dem Positivteil nichtnegativ sind, da der Abstand von $x$ zu den beiden Gitterpunkten $x_{\mathbf{i}_1}$ und$ x_{\mathbf{i}_2}$ kleiner gleich $\frac{2a}{M}$ ist. Zudem haben wir verwendet, dass $x_{\mathbf{i}_2} - x_{\mathbf{i}_1} = \frac{2a}{M}$ gilt, da beides Gitterpunkte sind.     
	
\emph{Induktionshypothese} (IH): Aussage \textbf{i)} gilt für eine beliebiges aber festes $d \in \N.$

\emph{Induktionsschritt} (IS): Wir nehmen oBdA. an, dass $x_{(0,\dots,0) \leq x \leq x_{(1,\dots,1}}$ gilt, mit $\mathbf{i}_1 = (0,\dots,0)$ und $\mathbf{i}_{(M + 1)}^{d + 1} = (1,\dots,1).$ Das heißt also, dass $x \in [-a, -a + \frac{2a}{M}]^{d + 1}$ gilt. Im folgenden zeigen wir $$\sum_{k = 1}^{(M + 1)^{d + 1}} \prod_{j = 1}^{d + 1} \bigg(1 - \frac{M}{2a} \cdot |x^{(j)} - x_{\mathbf{i}_k}^{(j)}|\bigg)_+ = 1.$$
Alle Summanden sind Null, wenn $|x^{(j)} - x_{\mathbf{i}_k}^{(j)}| \geq \frac{2a}{M}$ ist. Zudem haben wir oBdA angenommen dass $x \in [-a, -a + \frac{2a}{M}]^{d + 1}$ gilt, damit haben wir also nur noch $2^{d + 1}$ Summanden, nämlich die Anzahl der Gitterpunkte die am nähesten zu $x$ sind. Zudem wissen wir, dass alle Gitterpunkte, die in der $(d + 1)$ Komponente den selben Wert haben, sind in dieser Dimension gleich weit von $x^{(d + 1)}$ entfernt. Das heißt, in jedem Summanden kommt der Faktor $(1 - \frac{M}{2a} \cdot |x^{(d + 1)} - x_{(0,\dots,0)}^{(d + 1)}|)$ bzw. $(1 - \frac{M}{2a} \cdot |x^{(d + 1)} - x_{(1,\dots,1)}^{(d + 1)}|)$ vor, da 
\begin{equation*}
(1 - \frac{M}{2a} \cdot |x^{(d + 1)} - x_i^{(d + 1)}|) = \begin{cases}
(1 - \frac{M}{2a} \cdot |x^{(d + 1)} - x_{(0,\dots,0)}^{(d + 1)}|) &\text{$i \in \{0, 1\}^d \times \{0\}$}\\
(1 - \frac{M}{2a} \cdot |x^{(d + 1)} - x_{(1,\dots,1)}^{(d + 1)}|) &\text{$i \in \{0, 1\}^d \times \{1\}$}
\end{cases}
\end{equation*}
daraus ergibt sich$\colon$
\begin{equation*}
\begin{split}
\sum_{k = 1}^{(M + 1)^{d + 1}} & \prod_{j = 1}^{d + 1} \bigg(1 - \frac{M}{2a} \cdot |x^{(j)} - x_{\mathbf{i}_k}^{(j)}|\bigg)_+ \\
& = \sum_{i \in \{0, 1\}^{d + 1}} \prod_{j = 1}^{d + 1} \bigg(1 - \frac{M}{2a} \cdot |x^{(j)} - x_i^{(j)}|\bigg) \\
& = \Bigg(\sum_{i \in \{0, 1\}^d \times \{0\}} \prod_{j = 1}^{d} \bigg(1 - \frac{M}{2a} \cdot |x^{(j)} - x_i^{(j)}|\bigg)\Bigg) \cdot \bigg(1 - \frac{M}{2a} \cdot |x^{(d + 1)} - x_{(0,\dots,0)}^{(d + 1)}|\bigg) \\
& \qquad + \Bigg(\sum_{i \in \{0, 1\}^d \times \{1\}} \prod_{j = 1}^{d} \bigg(1 - \frac{M}{2a} \cdot |x^{(j)} - x_i^{(j)}|\bigg)\Bigg) \cdot \bigg(1 - \frac{M}{2a} \cdot |x^{(d + 1)} - x_{(1,\dots,1)}^{(d + 1)}|\bigg) \\
& \stackrel{(IV)}{=} 1 \cdot \bigg(1 - \frac{M}{2a} \cdot |x^{(d + 1)} - x_{(0,\dots,0)}^{(d + 1)}|\bigg) + 1 \cdot \bigg(1 - \frac{M}{2a} \cdot |x^{(d + 1)} - x_{(1,\dots,1)}^{(d + 1)}|\bigg) \\
& = 1 + 1 - \frac{M}{2a} \cdot |x^{(d + 1)} - x_{(0,\dots,0)}^{(d + 1)} + x_{(1,\dots,1)}^{(d + 1)} - x^{(d + 1)}| \\
& = 1 + 1 - 1 \\
& = 1,
\end{split}
\end{equation*}
wobei wir bei der vorletzten Gleichung angewendet haben, dass $x_{(1,\dots,1)} - x_{(0,\dots,0)} = \frac{2a}{M}$ ist, da beides Gitterpunkte sind.  $\hfill(\square)$ 		
		
\textbf{ii)} Es folgt $\prod_{j = 1}^d (1 - \frac{M}{2a} \cdot |x^{(j)} - x_{\mathbf{i}_k}^{(j)}|)_+ \geq 0$ für alle $k = 1,\dots, (M + 1)^d$, da $$z_+= \max\{z, 0\} \geq 0 (z \in \R)$$ gilt. Damit wäre die Nichtnegativität der Koeffizienten der Linearkombination gezeigt. Damit ist jeder Summand in $$\sum_{k = 1}^{(M + 1)^d} \prod_{j = 1}^d \bigg(1 - \frac{M}{2a} \cdot |x^{(j)} - x_{\mathbf{i}_k}^{(j)}|\bigg)_+$$ größer gleich Null und wegen i) muss dann auch $$\prod_{j = 1}^d \bigg(1 - \frac{M}{2a} \cdot |x^{(j)} - x_{\mathbf{i}_k}^{(j)}|\bigg)_+ \leq 1$$ gelten.

\textbf{iii)} Es handelt sich hierbei um eine lokale Konvexität, da die Bedingungen \textbf{i)} und \textbf{ii)} für alle $x \in [-a, a]$ gelten.
\end{proof}
Als nächstes zeigen wir ein Resultat für $(p, C)$-glatte Funktion welches wir im weiteren Verlauf dieser Arbeit wieder benötigen werden.
\begin{lem}
\label{lem:pcsmooth}
Sei $c_1$ eine Konstante. Es gelten die selben Voraussetzungen wie oben und $P(x)$ sei wie in \ref{konvexkomb} definiert, eine lokale Konvexkombination von Taylorpolynomen von $m$. Dann gilt für eine $(p, C)$-glatte Funktion $m\colon$
$$\sup_{x \in [-a, a]^d} |m(x) - P(x)|  \leq c_1 \cdot \frac{1}{M^p}.$$
\end{lem}
\begin{proof}
Nach dem Satz über die Lagrange Form des Restglieds (REFERENZ) existiert ein $\xi \in [x, x_{\mathbf{i}_k}],$ so, dass 
\begin{equation*}
\begin{split}
m(x) & = T_{x_{\mathbf{i}_k},q - 1}[m(x)] \\
& = \sum_{\substack{j_1,\dots,j_d \in \{0,\dots,q - 1\} \\j_1+\dots +j_d \leq q - 1}} \frac{1}{j_1! \cdots j_d!} \cdot \frac{\partial^{j_1+\cdots + j_d} m(x_{\mathbf{i}_k})}{\partial^{j_1} x^{(1)}\cdots \partial^{j_d} x^{(d)}} \cdot (x^{(1)} - x_{\mathbf{i}_k}^{(1)})^{j_1} \cdots (x^{(d)} - x_{\mathbf{i}_k}^{(d)})^{j_d} \\
 & \quad + \sum_{\substack{q - 1 < j_1,\dots,j_d \leq q \\j_1+\dots +j_d \leq q}} \frac{1}{j_1! \cdots j_d!} \cdot \frac{\partial^{j_1+\cdots + j_d} \xi}{\partial^{j_1} x^{(1)}\cdots \partial^{j_d} x^{(d)}}\cdot (x^{(1)} - x_{\mathbf{i}_k}^{(1)})^{j_1} \cdots (x^{(d)} - x_{\mathbf{i}_k}^{(d)})^{j_d}.
\end{split}
\end{equation*}
Nach dem Beweis von Lemma \ref{konvexkomb} \textbf{i)} erhalten wir 
$$m(x) = \sum_{k = 1}^{(M + 1)^d} m(x) \prod_{j = 1}^d \bigg(1 - \frac{M}{2a} \cdot |x^{(j)} - x_{\mathbf{i}_k}^{(j)}|\bigg)_+.$$ Zudem wissen wir dass man immer den Abstand zu den nähesten $2^d$ Gitterpunkten betrachtet, da  $(1 - \frac{M}{2a} \cdot |x^{(j)} - x_{\mathbf{i}_k}^{(j)}|)_+ = 0$ immer dann gilt, wenn der Abstand zwischen $x^{(j)}$ und $x_{\mathbf{i}_k}^{(j)}$ größer als $\frac{2a}{M}$ ist, daher ergibt sich$\colon$
\begin{equation*}
\begin{split}
& \sum_{\substack{q - 1 < j_1,\dots,j_d \leq q \\j_1+\dots +j_d \leq q}} \frac{1}{j_1! \cdots j_d!} \cdot \frac{\partial^{j_1+\cdots + j_d} \xi}{\partial^{j_1} x^{(1)}\cdots \partial^{j_d} x^{(d)}}\cdot (x^{(1)} - x_{\mathbf{i}_k}^{(1)})^{j_1} \cdots (x^{(d)} - x_{\mathbf{i}_k}^{(d)})^{j_d} \\
& \qquad \leq \sum_{\substack{q - 1 < j_1,\dots,j_d \leq q \\j_1+\dots +j_d \leq q}} \frac{1}{j_1! \cdots j_d!} \cdot \frac{\partial^{j_1+\cdots + j_d} \xi}{\partial^{j_1} x^{(1)}\cdots \partial^{j_d} x^{(d)}}\cdot \bigg(\frac{2a}{M}\bigg)^q
\end{split}
\end{equation*} 
und folgern damit und mithilfe der Dreiecksungleichung und der $(p,C)$-Glattheit von $m\colon$
\begin{equation*}
\begin{split}
|m(x) - P(x)| & \leq \sum_{k = 1}^{(M + 1)^d} |m(x) - p_{\mathbf{i}_k}| \prod_{j = 1}^d \bigg(1 - \frac{M}{2a} \cdot |x^{(j)} - x_{\mathbf{i}_k}^{(j)}|\bigg)_+ \\
& \leq \bigg(\frac{2a}{M}\bigg)^q \|\xi - x_{\mathbf{i}_k}\|^s \cdot C \cdot \sum_{k = 1}^{(M + 1)^d} \prod_{j = 1}^d \bigg(1 - \frac{M}{2a} \cdot |x^{(j)} - x_{\mathbf{i}_k}^{(j)}|\bigg)_+ \\
& = C \cdot \bigg(\frac{2a}{M}\bigg)^p \\
& = c_1 \cdot \frac{1}{M^p},
\end{split}
\end{equation*}
wobei wir bei der letzten Gleichung  Bedingung \textbf{i)} aus dem Beweis von Lemma \ref{konvexkomb} und $q + s = p$ verwendet haben.
\end{proof}

\section{Definition der Gewichte der Ausgabeschicht}

TBD