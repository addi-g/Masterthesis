\chapter{Konstruktion des Neuronale-Netze-Schätzers}
\label{chap:2}

In dieser Arbeit behandeln wir Neuronale-Netze-Regressionsschätzer im Kontext der \emph{nichtparametrischen Regression} mit \emph{zufälligem Design}. In der Regressionsanalyse betrachtet man einen Zufallsvektor $(X,Y)$, wobei $X$ $\R^d$-wertig und $Y$ reelwertig ist. Man ist daran interessiert, wie der Wert der \emph{Reaktionsvariable} $Y$ vom Wert des \emph{Beobachtungsvektors} $X$ abhängt. Dies bedeutet, dass man eine (messbare) Funktion $f\colon \R^d \to \R$ sucht, sodass $f(X)$ eine \glqq gute Approximation von $Y$ ist \grqq. Das wiederum bedeutet, dass $f(X)$ nah an $Y$ sein sollte, was in gewisser Weise gleichwertig ist, den Ausdruck $|f(X) - Y|$ \glqq klein\grqq zu machen. Da aber $X$ und $Y$ Zufallsvektoren sind und damit $|f(X) - Y|$ auch zufällig ist, ist nicht klat was mit unter \glqq $|f(X) - Y|$ klein\grqq zu verstehen ist. Wir können dieses Problem lösen, indem wir das sogenannte $L_2$-Risiko 
$$
\E[|f(X) - Y|^2],
$$
einführen und verlangt, dass es so klein wie möglich sein muss.
Bei der nichtparametrischen Regression ist die Bauart der zu schätzenden Funktion komplett unbekannt, was den Vorteil besitzt, dass weniger Annahmen getroffen werden müssen, man aber dadurch noch mehr Daten benötigt, um eine Funktion zu schätzen.
Das Problem der Regressionsschätzung bei zufälligem Design lässt sich wie folgt erläutern: 
In Anwendungsfällen ist üblicherweise die Verteilung von $(X, Y)$ unbekannt, daher kann die \emph{Regressionsfunktion} $m(x) = \E[Y \mid X = x]$ nicht berechnet werden. Oft ist es aber möglich, Werte von $(X, Y)$ zu beobachten.
Für das $L_2$-Risiko einer beliebigen messbaren Funktion $f\colon \R^d \to \R$  gilt:
$$\E[|f(X) - Y|^2] = \E[|m(X) - Y|^2] + \int|f(x) - m(x)|^2 \mathds{P}_X (dx),$$
d.h. der mittlere quadratische Vorhersagefehler einer Funktion ist darstellbar als Summe des $L_2$-Risikos der Regressionsfunktion (unvermeidbarer Fehler) und des $L_2$-Fehlers, der aufgrund der Verwendung von $f$ an Stelle von $m$ bei der Vorhersage bzw.\@ Approximation des Wertes von Y entsteht.
Im Hinblick auf die Minimierung des \emph{$L_2$-Risikos} sollte dabei der $L_2$-Fehler der Schätzfunktion möglichst klein sein.
Formal führt das daher auf folgende Problemstellung:
\begin{prblm}
$(X, Y), (X_1, Y_1), (X_2, Y_2), \dots$ seien u.\@i.\@v.\@ $\R^d \times \R$ wertige Zufallsvariablen mit $\E[Y^2] < \infty$ und $m\colon\R^d \to \R$ definiert durch $m(x) = \E[Y \mid X = x]$ sei die zugehörige Regressionsfunktion. Gegeben sei die Datenmenge 
\begin{equation}
\label{dataset}
\mathcal{D}_n = \{(X_1, Y_1),\dots,(X_n, Y_n)\}.
\end{equation}
Gesucht ist eine Schätzung 
$$m_n(\cdot) = m_n(\cdot, \mathcal{D}_n)\colon\R^d \to \R $$
von $m$, für die der $L_2$-Fehler 
$$\int |m_n(x) - m(x)|^2 \mathds{P}_X(dx)$$
möglichst \glqq klein\grqq\@ ist \cite[Kapitel 1.1 und Kapitel 1.2]{gyoerfi2002}. 
\end{prblm}
In diesem Kapitel werden wir mithilfe der gegebenen Datenmenge $\mathcal{D}_n$ unseren Regressionsschätzer $\tilde{m}_n$ konstruieren. 
In Kapitel~\ref{chap:1} haben wir bereits in Definition~\ref{def:nn} vorgestellt, was wir unter einem mehrschichtigen feedforward neuronalen Netz mit Architektur $(L, \bk)$ und Aktivierungsfunktion $\sigma$ verstehen.

Für die Konstruktion unseres Neuronale-Netze-Schätzers wählen wir den logistischen Squasher~(\ref{logsquasher}) als Aktivierungsfunktion $\sigma$, verwenden die gegebene Datenmenge $\mathcal{D}_n$ und wählen die Gewichte des neuronalen Netzes so, dass die resultierende Funktion aus Definition~\ref{def:nn} eine gute Schätzung für die Regressionsfunktion $m$ ist. Dafür wählen wir die Gewichte bis auf die in der Ausgabeschicht fest und schätzen die Gewichte in der Ausgabeschicht, indem wir mit unserer Datenmenge~(\ref{dataset}) ein Kleinste-Quadrate-Problem lösen.
%mit einer Tikhonov Regularisierung \cite[Theorem 5.9]{Kress2012} lösen.

Es ist bekannt (vgl.\@ \cite[Theorem 7.2 und Problem 7.2]{DevLug96} und \cite[Section 3]{DevWag80}), dass man Glattheitsvoraussetzungen an die Regressionsfunktion stellen muss, um nichttriviale Konvergenzresultate für nichtparametrische Regressionsschätzer herzuleiten. Dafür verwenden wir die folgende Definition.
\begin{defn}[($p,C$)-Glattheit]
\label{def:pc}
   Sei $p = q + s$ mit $q \in \N_0$ und $s \in (0,1]$ (also $p \in (0, \infty)$ und sei $C > 0$. Eine Funktion $f\colon \R^d \to \R$ heißt \emph{($p, C$)-glatt}, falls für alle $\alpha = (\alpha_1,\dots,\alpha_d) \in \N_0^d$ mit $\sum_{j = 1}^{d}\alpha_j = q$ die partielle Ableitung 
   $$ \frac{\partial^qf}{\partial x_1^{\alpha_1}\dots\partial x_d^{\alpha_d}}$$
   existiert und falls für alle $x, z \in \R^d$ die Abschätzung 
   $$ \bigg|\frac{\partial^qf}{\partial x_1^{\alpha_1}\dots\partial x_d^{\alpha_d}}(x) - \frac{\partial^qf}{\partial x_1^{\alpha_1}\dots\partial x_d^{\alpha_d}}(z) \bigg| \leq C \cdot \|x - z\|^s,$$
   gilt, wobei $\|\cdot\|$ die euklidische Norm in $\R^d$ bezeichnet.  
\end{defn}

\section{Definition der Netzwerkarchitektur}
\label{subsec:2:1}
In diesem Abschnitt stellen wir die \emph{Netzwerkachitektur} unsere Neuronale-Netze-Regressionsschätzers vor. Dafür legen wir die Architektur $(L,\bk)$ fest und gehen auf die konkrete Konstruktion unseres Schätzers ein.

Zunächst fixieren wir die Multiindexnotation, die wir der Übersichtlichkeit halber im weiteren Verlauf dieser Arbeit verwenden werden. Sei $a > 0$ fest und $M \in \N$. 

Sei $[M]^d \coloneqq\{0, 1, \dots, M\}^d.$ 
Für $(\mathbf{i}^{(1)},\dots,\mathbf{i}^{(d)}) = \bi \in [M]^d$ und $x \in \R^d$ definieren wir
$$|\mathbf{i}|_1 \coloneqq \sum_{k= 1}^d \mathbf{i}^{(k)} \text{, } \quad \mathbf{i}! \coloneqq \mathbf{i}^{(1)}! \cdots \, \mathbf{i}^{(d)}! \quad \text{ und } \quad x^{\mathbf{i}} \coloneqq x_1^{\mathbf{i}^{(1)}} \cdots \,    x_d^{\mathbf{i}^{(d)}}.$$
Für $f\colon \R^d \to \R$ ausreichend oft differenzierbar definieren wir 
$$\partial^{\mathbf{i}}f(x) \coloneqq \frac{\partial^{|\mathbf{i}|_1}f}{\partial^{\mathbf{i}^{(1)}} x_1 \cdots \, \partial^{\mathbf{i}^{(d)}} x_d} (x).$$
Wir betrachten im Folgenden ein $d$-dimensionales äquidistantes Gitter im Würfel $[-a, a]^d$ mit Schrittweite $\frac{2a}{M}.$ 
%Sei $\bi_1,\dots,\bi_{(M + 1)^d}$ eine Aufzählung der Elemente von $[M]^d$.
Dann ordnen wir jedem Multiindex $\bi \in [M]^d$ einen Gitterpunkt
\begin{equation}
\label{eq:gitter}
x_{\bi} = \bigg( -a + \bi^{(1)} \cdot \frac{2a}{M},\dots, -a + \bi^{(d)} \cdot \frac{2a}{M}\bigg) = -\mathbf{a} + \frac{2a}{M} \cdot \bi,
\end{equation}
zu, mit $\mathbf{a} = (a, a, \dots, a) \in \R^d$.

Hiermit lässt sich das zu $m$ gehörige Taylorpolynom der Ordnung $q \in \N_0$ mit Entwicklungspunkt $x_{\bi}$ schreiben als
$$p_{\bi}^m(x) = \sum_{\substack{\mathbf{j} \in [q]^d \\|\mathbf{j}|_1 \leq q}} \partial^{\mathbf{j}}m(x_{\mathbf{i}}) \cdot \frac{(x - x_{\mathbf{i}})^{\mathbf{j}}}{\mathbf{j}!}.$$
Zudem betrachten wir eine Funktion
\begin{equation}
\label{konvexkomb}
P_m(x) = \sum_{\bi \in [M]^d} p_{\bi}^m(x) \prod_{j = 1}^{d} \bigg(1 - \frac{M}{2a} \cdot |x^{(j)} - x_{\mathbf{i}}^{(j)}|\bigg)_+,
\end{equation}
für die wir im weiteren Verlauf dieser Arbeit zeigen werden, dass diese die Regeressionsfunktion $m$ approximiert.

Wir zeigen mithilfe des folgenden Lemmas, dass $P_m(x)$ eine lokale Spline Interpolation von Taylorpolynomen von $m$ ist.
\begin{lem}
\label{lem:loccon}
Sei $a >0$ und $M \in \N$. Dann sind für $\bi \in [M]^d$ die Funktionen 
$$B_{\bi}(x) = \prod_{j = 1}^{d} \bigg(1 - \frac{M}{2a} \cdot |x^{(j)} - x_{\mathbf{i}}^{(j)}|\bigg)_+ \text{ für } x \in [-a, a]^d,$$
mit $x_{\bi}$ als Gitterpunkte aus Gleichung~(\ref{eq:gitter}), B-Splines auf $[-a, a]^d$, für die die folgenden drei Bedingungen gelten:
\begin{itemize}
\item[\textbf{i)}] \emph{Zerlegung der Eins}: $\sum_{\bi \in [M]^d} B_{\bi}(x) = 1$ für $x \in [-a, a]$.
\item[\textbf{ii)}] \emph{Nichtnegativität}: $B_{\bi}(x) \geq 0$ für alle $\bi \in [M]^d$.
\item[\textbf{iii)}] \emph{Lokaler Träger}: Für $\bi \in [M]^d$ ist $B_{\bi}(x) > 0$ falls $|x^{(j)} - x_{\bi}^{(j)}| < \frac{2a}{M}$ für alle $j \in \{1,\dots,d\}$ gilt und andernfalls $B_{\bi}(x) = 0$.
\end{itemize}
\end{lem}
\begin{proof}
Als erstes möchten wir für $d = 2$ und $M = 3$ eine Skizze angeben, um die Idee des Beweises zu veranschaulichen. 
\begin{figure}[htp]
\centering
\begin{tikzpicture} 
   %Raster zeichnen 
   %\draw [color=gray!50]  [step=20mm] (-3,-3) grid (4,4); 
   \draw [color=gray!50] (0,0) -- (6,0) -- (6,6) -- (0,6) -- (0,0);			
   \draw [color=gray!50] (0,2) -- (6,2);
   \draw [color=gray!50] (0,4) -- (6,4);
        \draw [color=gray!50] (2,6) -- (2,0);
                \draw [color=gray!50] (4,6) -- (4,0);

	\fill[black] (0,0) circle (0.08cm) node[label=below:{$(-a, -a)$}]{};
		\fill[black] (6,0) circle (0.08cm) node[label=below:{$(a, -a)$}]{};
				\fill[black] (0,6) circle (0.08cm) node[label=above:{$(-a, a)$}]{};
						\fill[black] (6,6) circle (0.08cm) node[label=above:{$(a, a)$}]{};
						\fill[black] (0,0) circle (0.00cm) node[label=left:{$x_{\mathbf{i}_1}$}]{};
						\fill[black] (2,0) circle (0.00cm) node[label=below:{$x_{\mathbf{i}_2}$}]{};
						\fill[black] (4,0) circle (0.00cm) node[label=below:{$x_{\mathbf{i}_3}$}]{};
						\fill[black] (6,0) circle (0.00cm) node[label=right:{$x_{\mathbf{i}_4}$}]{};

						\fill[black] (9,6) circle (0.00cm) node[label=right:{$\mathbf{i}_1 = (0, 0)$}]{};			
						\fill[black] (9,5.5) circle (0.00cm) node[label=right:{$\mathbf{i}_2 = (1, 0)$}]{};					
						\fill[black] (9,5) circle (0.00cm) node[label=right:{$\mathbf{i}_3 = (2, 0)$}]{};	
						\fill[black] (9,4.5) circle (0.00cm) node[label=right:{$\mathbf{i}_4 = (3, 0)$}]{};	
						\draw [dotted, ultra thick] (10,4) -- (10,3.65);
						\fill[black] (9,3.15) circle (0.00cm) node[label=right:{$\mathbf{i}_{16} = (3, 3)$}]{};	
						
						\fill[black] (0,2) circle (0.08cm);
						\fill[black] (0,4) circle (0.08cm); 
						\fill[black] (2,2) circle (0.08cm);
						\fill[black] (2,4) circle (0.08cm);
						\fill[black] (2,6) circle (0.08cm);
						\fill[black] (2,0) circle (0.08cm);
						\fill[black] (0,4) circle (0.08cm);
						\fill[black] (2,4) circle (0.08cm);
						\fill[black] (4,4) circle (0.08cm);
						\fill[black] (4,6) circle (0.08cm);																\fill[black] (4,0) circle (0.08cm);
						\fill[black] (4,2) circle (0.08cm);						
						\fill[black] (6,2) circle (0.08cm);
						\fill[black] (6,4) circle (0.08cm);
\end{tikzpicture}
\caption{Beispielhafte Darstellung der $x_{\mathbf{i}_k}$ für $d = 2$ und $M = 3$.}
\label{fig:gitter}
\end{figure}
Es ist ein Gitter mit $(M + 1)^d$ Gitterpunkten, die den $x_{\mathbf{i}_k}$ entsprechen. Der Abstand zwischen zwei Gitterpunkten beträgt $\frac{2a}{M}.$ Man betrachtet immer den Abstand zu den nächsten $2^d$ Gitterpunkten, da  $(1 - \frac{M}{2a} \cdot |x^{(j)} - x_{\mathbf{i}}^{(j)}|)_+ = 0$ immer dann gilt, wenn der Abstand zwischen $x^{(j)}$ und $x_{\mathbf{i}}^{(j)}$ größer als $\frac{2a}{M}$ ist.     

\emph{\textbf{i)}} Im Folgenden wollen wir 
\begin{equation}
\label{induktiongl1}
\sum_{\bi \in [M]^d} B_{\bi}(x) = 1 \text{ für } x \in [-a , a]^d
\end{equation}
per Induktion über $d$ zeigen. 
	
	 \emph{Induktionsanfang} (IA): Für $d = 1$ kann $x$ nur zwischen zwei Gitterpunkten $x_{\mathbf{i}_1} \neq x_{\mathbf{i}_2}$ liegen. Sei ohne Beschränkung der Allgemeinheit $x_{\mathbf{i}_1}\leq x \leq x_{\mathbf{i}_2}$, dann gilt mit der gleichen Begründung wie im einleitenden Beispiel:
	\begin{equation*}
	\begin{split}
	\sum_{\bi \in [M]^d} (1 - \frac{M}{2a} \cdot |x - x_{\mathbf{i}}|)_+ & = (1 - \frac{M}{2a} \cdot |x - x_{\mathbf{i}_1}|)_+ + (1 - \frac{M}{2a} \cdot |x - x_{\mathbf{i}_2}|)_+ \\
	& = 1 + 1 - \frac{M}{2a} \cdot (x - x_{\mathbf{i}_1} + x_{\mathbf{i}_2} - x) \\
	& = 1 + 1 - \frac{M}{2a} \cdot \frac{2a}{M} \\
	& = 1,
	\end{split}
	\end{equation*} wobei wir unter anderem verwendet haben, dass beide Summanden unabhängig von dem Positivteil nichtnegativ sind, da der Abstand von $x$ zu den beiden Gitterpunkten $x_{\mathbf{i}_1}$ und $x_{\mathbf{i}_2}$ kleiner gleich $\frac{2a}{M}$ ist. Zudem haben wir verwendet, dass $x_{\mathbf{i}_2} - x_{\mathbf{i}_1} = \frac{2a}{M}$ gilt, da beides Gitterpunkte sind.     
	
\emph{Induktionshypothese} (IH): Aussage~(\ref{induktiongl1}) gelte für ein beliebiges aber festes $d \in \N.$

\emph{Induktionsschritt} (IS): Wir nehmen ohne Beschränkung der Allgemeinheit an, dass $x_{(0,\dots,0)} \leq x \leq x_{(1,\dots,1)}$ komponentenweise gilt. Das heißt also, dass $x \in [-a, -a + \frac{2a}{M}]^{d + 1}$ gilt. Im Folgenden zeigen wir $$\sum_{\bi \in [M]^{(d + 1)}} B_{\bi}(x) = \sum_{\bi \in [M]^{(d + 1)}}\prod_{j = 1}^{d + 1} \bigg(1 - \frac{M}{2a} \cdot \big|x^{(j)} - x_{\mathbf{i}}^{(j)}\big|\bigg)_+ = 1.$$
Ein Summand ist Null, wenn ein $j \in \{1,\dots,d+1\}$ existiert mit $\big|x^{(j)} - x_{\mathbf{i}}^{(j)}\big| \geq \frac{2a}{M}$. Zudem haben wir ohne Beschränkung der Allgemeinheit angenommen, dass $x \in [-a, -a + \frac{2a}{M}]^{d + 1}$ gilt. Damit haben wir also nur noch $2^{d + 1}$ Summanden, was der Anzahl der Gitterpunkte die am nächsten bei $x$ liegen entspricht. Zudem wissen wir, dass alle Gitterpunkte, die in der $(d + 1)$-ten Komponente den selben Wert haben, in dieser Dimension gleich weit von $x^{(d + 1)}$ entfernt sind. Das heißt, in jedem Summanden kommt der Faktor $(1 - \frac{M}{2a} \cdot \big|x^{(d + 1)} - x_{(0,\dots,0)}^{(d + 1)}\big|)$ bzw.\@ $(1 - \frac{M}{2a} \cdot \big|x^{(d + 1)} - x_{(1,\dots,1)}^{(d + 1)}\big|)$ vor, da 
\begin{equation*}
\bigg(1 - \frac{M}{2a} \cdot \Big|x^{(d + 1)} - x_\mathbf{i}^{(d + 1)}\Big|\bigg) = \begin{cases}
(1 - \frac{M}{2a} \cdot \big|x^{(d + 1)} - x_{(0,\dots,0)}^{(d + 1)}\big|) &\text{$\mathbf{i} \in \{0, 1\}^d \times \{0\}$}\\
(1 - \frac{M}{2a} \cdot |x^{(d + 1)} - x_{(1,\dots,1)}^{(d + 1)}|) &\text{$\mathbf{i} \in \{0, 1\}^d \times \{1\}$}
\end{cases}
\end{equation*}
gilt. Daraus ergibt sich:
\begin{equation*}
\begin{split}
\sum_{\bi \in [M]^{(d + 1)}} & \prod_{j = 1}^{d + 1} \bigg(1 - \frac{M}{2a} \cdot |x^{(j)} - x_{\mathbf{i}}^{(j)}|\bigg)_+ \\
& = \sum_{\mathbf{i} \in \{0, 1\}^{d + 1}} \prod_{j = 1}^{d + 1} \bigg(1 - \frac{M}{2a} \cdot |x^{(j)} - x_{\bi}^{(j)}|\bigg) \\
& = \Bigg(\sum_{\mathbf{i} \in \{0, 1\}^d \times \{0\}} \prod_{j = 1}^{d} \bigg(1 - \frac{M}{2a} \cdot |x^{(j)} - x_{\bi}^{(j)}|\bigg)\Bigg) \cdot \bigg(1 - \frac{M}{2a} \cdot |x^{(d + 1)} - x_{(0,\dots,0)}^{(d + 1)}|\bigg) \\
& \qquad + \Bigg(\sum_{\mathbf{i} \in \{0, 1\}^d \times \{1\}} \prod_{j = 1}^{d} \bigg(1 - \frac{M}{2a} \cdot |x^{(j)} - x_{\bi}^{(j)}|\bigg)\Bigg) \cdot \bigg(1 - \frac{M}{2a} \cdot |x^{(d + 1)} - x_{(1,\dots,1)}^{(d + 1)}|\bigg) \\
& \stackrel{(\text{IV})}{=} 1 \cdot \bigg(1 - \frac{M}{2a} \cdot \Big|x^{(d + 1)} - x_{(0,\dots,0)}^{(d + 1)}\Big|\bigg) + 1 \cdot \bigg(1 - \frac{M}{2a} \cdot \Big|x^{(d + 1)} - x_{(1,\dots,1)}^{(d + 1)}\Big|\bigg) \\
& = 1 + 1 - \frac{M}{2a} \cdot \Big|x^{(d + 1)} - x_{(0,\dots,0)}^{(d + 1)} + x_{(1,\dots,1)}^{(d + 1)} - x^{(d + 1)}\Big| \\
& = 1 + 1 - 1 \\
& = 1,
\end{split}
\end{equation*}
wobei wir bei der vorletzten Gleichung angewendet haben, dass $x_{(1,\dots,1)}^{(d + 1)} - x_{(0,\dots,0)}^{(d + 1)} = \frac{2a}{M}$ ist, da beides Gitterpunkte sind.  $\hfill(\square)$ 		

\emph{\textbf{ii)}} Es folgt $\prod_{j = 1}^d (1 - \frac{M}{2a} \cdot |x^{(j)} - x_{\mathbf{i}}^{(j)}|)_+ \geq 0$ für alle $\bi \in [M]^d$, da $$z_+= \max\{z, 0\} \geq 0 \text{ für $z \in \R$}$$ gilt.

\emph{\textbf{iii)}} Es handelt sich hierbei um einen lokalen Träger, da nach der Konstruktion von $B_{\bi}(x)$ der Funktionswert genau dann Null ist, wenn ein $j \in \{1,\dots, d\}$ existiert, sodass $|x^{(j)} - x_{\bi}^{(j)}| \geq \frac{2a}{M}$ gilt. Andernfalls erhalten wir mit Bedingung~\emph{\textbf{ii)}}, dass $B_{\bi}(x) > 0$ ist. 
\end{proof}

Mit der Definition eines B-Splines aus Lemma~\ref{lem:loccon} erhalten wir nun:
\begin{equation}
\label{def:Pm}
\begin{split}
P_m(x) &= \sum_{\bi \in [M]^d} p_{\bi}^m(x) \prod_{j = 1}^{d} \bigg(1 - \frac{M}{2a} \cdot |x^{(j)} - x_{\mathbf{i}}^{(j)}|\bigg)_+ \\
& = \sum_{\bi \in [M]^d} p_{\bi}^m(x) \cdot B_{\bi}(x).
\end{split}
\end{equation}
Damit wissen wir, dass $P_m(x)$ eine Spline Interpolation von Taylorpolynomen von $m$ ist.
Die Wahl der Architektur $(L,\bk)$ unseres Neuronale-Netze-Regressionsschätzers und der Werte aller Gewichte bis auf die der Ausgabeschicht ist durch folgendes Approximationsresultat motiviert.
\begin{lem}
\label{lem:pcsmooth}
Sei $M \in \N$, $a > 0$ und $f$ eine ($p, C$)-glatte Funktion, wobei $p = q + s$ mit $q \in \N_0, s \in (0,1]$ und $C > 0$ sind. Sei zudem $P_f(x)$ analog zu Gleichung~(\ref{def:Pm}) eine lokale Spline Interpolation von Taylorpolynomen von $f$. Dann gilt:
$$\sup_{x \in [-a, a]^d} |f(x) - P_f(x)|  \leq c \cdot \bigg(\frac{a}{M}\bigg)^p,$$
mit einer Konstante $c$, die von $p, d, s$ und $C$ abhängt.
\end{lem}
\begin{proof}
Nach Lemma~\ref{lem:lagrange} \textbf{i)} über die Lagrange Form des Restglieds existiert ein $\xi$ auf der Verbindungsstrecke zwischen $x_{\bi}$ und $x$, so, dass 
\begin{equation}
\label{eq:lagrange}
\begin{split}
f(x) & = T_{q - 1}f(x; x_{\mathbf{i}}) + R_q(x; x_{\mathbf{i}})\\
& = \sum_{\substack{\bj \in [q]^d \\ |\bj|_1 \leq q - 1}}  \partial^{\bj}f(x_{\mathbf{i}}) \cdot \frac{ (x - x_{\mathbf{i}})^{\bj} }{\bj!} + \sum_{\substack{ \bj \in [q]^d \\|\bj|_1 = q}} \partial^{\bj}f(\xi) \frac{ (x - x_{\mathbf{i}})^{\bj} }{\bj!}.
\end{split}
\end{equation}
Nach der B-Spline Eigenschaft aus Gleichung~(\ref{induktiongl1}) erhalten wir 
$$f(x) = \sum_{\bi \in [M]^d} f(x) \prod_{j = 1}^d \bigg(1 - \frac{M}{2a} \cdot |x^{(j)} - x_{\mathbf{i}}^{(j)}|\bigg)_+.$$ Zudem wissen wir, dass man immer den Abstand zu den nächsten $2^d$ Gitterpunkten betrachtet, da  $(1 - \frac{M}{2a} \cdot |x^{(j)} - x_{\mathbf{i}}^{(j)}|)_+ = 0$ immer dann gilt, wenn der Abstand zwischen $x^{(j)}$ und $x_{\mathbf{i}}^{(j)}$ größer als $\frac{2a}{M}$ ist. Daher ergibt sich:
\begin{equation}
\label{eq:absch}
\begin{split}
& \sum_{\substack{ \bj \in [q]^d \\ |\bj|_1 = q}} \partial^{\bj}f(\xi) \frac{ (x - x_{\mathbf{i}})^{\bj} }{\bj!} \leq \sum_{\substack{ \bj \in [q]^d \\ |\bj|_1 = q}} \partial^{\bj}f(\xi) \frac{1}{\bj!} \cdot \bigg(\frac{2a}{M}\bigg)^q.
\end{split}
\end{equation} 
Mithilfe der Dreiecksungleichung und der Konstruktion von $P_f(x)$ erhalten wir:
\begin{equation}
\label{eq:drei}
\begin{split}
|f(x) - P_f(x)| \leq \sum_{\bi \in [M]^d} \bigg|f(x) - \sum_{\substack{\mathbf{j} \in [q]^d \\|\mathbf{j}|_1 \leq q}} \partial^{\mathbf{j}}f(x_{\mathbf{i}}) \cdot \frac{(x - x_{\mathbf{i}})^{\mathbf{j}}}{\mathbf{j}!}\bigg| \prod_{j = 1}^d \bigg(1 - \frac{M}{2a} \cdot |x^{(j)} - x_{\mathbf{i}}^{(j)}|\bigg)_+.
\end{split}
\end{equation}
Nach Gleichung~(\ref{eq:lagrange}) erhalten wir:
\begin{equation}
\label{eq:sum}
\begin{split}
& \bigg|f(x) - \sum_{\substack{\mathbf{j} \in [q]^d \\|\mathbf{j}|_1 \leq q}} \partial^{\mathbf{j}}f(x_{\mathbf{i}}) \cdot \frac{(x - x_{\mathbf{i}})^{\mathbf{j}}}{\mathbf{j}!}\bigg| \\ 
& = \bigg| \sum_{\substack{\bj \in [q]^d \\ |\bj|_1 \leq q - 1}}  \partial^{\bj}f(x_{\mathbf{i}}) \cdot \frac{ (x - x_{\mathbf{i}})^{\bj} }{\bj!} + \sum_{\substack{ \bj \in [q]^d \\|\bj|_1 = q}} \partial^{\bj}f(\xi) \frac{ (x - x_{\mathbf{i}})^{\bj} }{\bj!} \\
& \qquad \qquad \qquad \qquad \qquad \qquad \qquad \qquad \qquad - \sum_{\substack{\mathbf{j} \in [q]^d \\|\mathbf{j}|_1 \leq q}} \partial^{\mathbf{j}}f(x_{\mathbf{i}}) \cdot \frac{(x - x_{\mathbf{i}})^{\mathbf{j}}}{\mathbf{j}!}\bigg| \\
& = \bigg| \sum_{\substack{ \bj \in [q]^d \\|\bj|_1 = q}} \partial^{\bj}f(\xi) \frac{ (x - x_{\mathbf{i}})^{\bj} }{\bj!} - \sum_{\substack{ \bj \in [q]^d \\|\bj|_1 = q}} \partial^{\bj}f(x_{\bi}) \frac{ (x - x_{\mathbf{i}})^{\bj} }{\bj!}\bigg|.
\end{split}
\end{equation}
Aus der $(p,C)$-Glattheit von $f$ mit $\xi, x_{\bi} \in \R^d$ folgt durch (\ref{eq:absch}) und (\ref{eq:sum}):
\begin{equation}
\label{eq:last}
\begin{split}
& \bigg| \sum_{\substack{ \bj \in [q]^d \\|\bj|_1 = q}} \partial^{\bj}f(\xi) \frac{ (x - x_{\mathbf{i}})^{\bj} }{\bj!} - \sum_{\substack{ \bj \in [q]^d \\|\bj|_1 = q}} \partial^{\bj}f(x_{\bi}) \frac{ (x - x_{\mathbf{i}})^{\bj} }{\bj!}\bigg| \\
& \leq \bigg(\frac{2a}{M}\bigg)^q \|\xi - x_{\bi} \|^s \cdot C.
\end{split}
\end{equation}
Fassen wir die Gleichungen~(\ref{eq:drei}), (\ref{eq:sum}) und (\ref{eq:last}) zusammen, erhalten wir:
\begin{equation*}
\begin{split}
|f(x) - P_f(x)| & \leq \sum_{\bi \in [M]^d} \bigg|f(x) - \sum_{\substack{\mathbf{j} \in [q]^d \\|\mathbf{j}|_1 \leq q}} \partial^{\mathbf{j}}f(x_{\mathbf{i}}) \cdot \frac{(x - x_{\mathbf{i}})^{\mathbf{j}}}{\mathbf{j}!}\bigg| \prod_{j = 1}^d \bigg(1 - \frac{M}{2a} \cdot |x^{(j)} - x_{\mathbf{i}}^{(j)}|\bigg)_+ \\
& \leq \bigg(\frac{2a}{M}\bigg)^q \|\xi - x_{\bi} \|^s \cdot C \cdot \sum_{\bi \in [M]^d} \prod_{j = 1}^d \bigg(1 - \frac{M}{2a} \cdot |x^{(j)} - x_{\mathbf{i}}^{(j)}|\bigg)_+ \\
& \leq \bigg(\frac{2a}{M}\bigg)^q \cdot \bigg(\frac{2a}{M}\bigg)^s \cdot d^{s/2} \cdot C\\
& = c \cdot \bigg(\frac{a}{M}\bigg)^p,
\end{split}
\end{equation*}
wobei wir bei der letzten Ungleichung die Eingenschaft der Zerlegung der Eins aus Gleichung~(\ref{induktiongl1}) und $q + s = p$ verwendet haben, mit 
$$c = 2^p \cdot d^{s/2} \cdot C.$$
Bilden wir schließlich noch das Supremum über $x \in [-a, a]^d$, erhalten wir die Behauptung.
\end{proof}

Durch geeignet gewählte $a_{\bi, \bj} \in \R$ lässt sich $P_m(x)$ in die Form 
$$\sum_{\bi \in [M]^d} \sum_{\substack{ \bj \in [q]^d\\|\bj|_1 \leq q}} a_{\bi, \bj} \cdot (x - x_{\mathbf{i}})^{\bj} \prod_{j = 1}^d \bigg(1 - \frac{M}{2a} \cdot |x^{(j)} - x_{\mathbf{i}}^{(j)}|\bigg)_+$$
bringen, da sich jedes $p^m_{\bi}(x)$ als Polynom umordnen lässt und wir daher auch $P_m(x)$ umschreiben können.

Als Nächstes wollen wir geeignete neuronale Netze $f_{\net, \bj, \bi}$ mit Architektur $(L, \bk)$ definieren, die die Funktionen
$$x \mapsto (x - x_{\mathbf{i}})^{\bj} \prod_{j = 1}^d \bigg(1 - \frac{M}{2a} \cdot |x^{(j)} - x_{\mathbf{i}}^{(j)}|\bigg)_+$$ approximieren. Zudem möchten wir die Architektur $(L, \bk)$ so wählen, dass neuronale Netze der Form
$$\sum_{\bi \in [M]^d} \sum_{\substack{ \bj \in [q]^d\\|\bj|_1 \leq q}} a_{\mathbf{i}, \bj} \cdot f_{\net,\bj,\mathbf{i}}(x) \qquad (a_{\mathbf{i},\bj} \in \R)$$ in der Klasse $\mathfrak{N}(L, \bk, \sigma)$ enthalten sind.
Um dies zu erreichen, sei $$\sigma(x) = \frac{1}{(1 + \exp(-x))} \quad (x \in \R)$$ der logistische Squasher~(\ref{logsquasher}). Zudem wählen wir $R \geq 1$ und definieren die folgenden neuronalen Netze:

Das neuronale Netz
\begin{equation}
\label{def:fid}
f_{\id}(x) = 4R \cdot \sigma\Big(\frac{x}{R}\Big) - 2R,
\end{equation}
mit Architektur $(1, (1))$, welches, wie in Lemma~\ref{lem:1} gezeigt, die Funktion $f(x) = x$ approximiert und in Abbildung~\ref{fig:fid} veranschaulicht wurde.

Das neuronale Netz 
\begin{equation}
\label{def:fmult}
\begin{split}
f_{\mult}(x, y) = \frac{R^2}{4} \cdot \frac{(1 + \exp(- 1))^3}{\exp(-2) - \exp(-1)} \cdot & \bigg(\sigma\Big(\frac{2(x + y)}{R} + 1\Big) - 2 \cdot \sigma \Big(\frac{x + y}{R} + 1\Big) \\
& - \sigma\Big(\frac{2(x - y)}{R} + 1\Big) + 2 \cdot \sigma\Big(\frac{x - y}{R} + 1\Big)\bigg),
\end{split}
\end{equation}
mit Architektur $(1, (4))$, welches, wie in Lemma~\ref{lem:2} gezeigt, die Funktion $f(x, y) = x \cdot y$ approximiert und in Abbildung~\ref{fig:fmult} veranschaulicht wurde.

Das neuronale Netz 
\begin{equation}
\label{def:frelu}
f_{\ReLU}(x) = f_{\mult}(f_{\id}(x), \sigma(R \cdot x)),
\end{equation}
mit Architektur $(2, (2, 4))$, welches, wie in Lemma~\ref{lem:3} gezeigt, die Funktion $f(x) = x_+$ approximiert und in Abbildung~\ref{fig:frelu} veranschaulicht wurde und schließlich das neuronale Netz 
\begin{equation}
\label{def:fhat}
f_{\mathrm{hat},y}(x) = f_{\ReLU}\bigg(\frac{M}{2a} \cdot (x - y) + 1\bigg) - 2 \cdot f_{\ReLU}\bigg(\frac{M}{2a} \cdot (x - y)\bigg) +  f_{\ReLU}\bigg(\frac{M}{2a} \cdot (x - y) - 1\bigg),
\end{equation}
mit Architektur $(2, (6, 12))$, welches, wie in Lemma~\ref{lem:4} gezeigt, für fixes $y \in \R$ die Funktion $$f(x) = \bigg(1 - \bigg(\frac{M}{2a}\bigg) \cdot |x - y|\bigg)_+$$ approximiert und in Abbildung~\ref{fig:fhat} veranschaulicht wurde.

Mit diesen neuronalen Netzen können wir nun $f_{\net,\bj,\bi}$ rekursiv definieren. Dafür wählen wir $N > q$, setzen $s = \lceil\log_2(N + d)\rceil$ und definieren für $\bj \in [N]^d$, $\bi \in [M]^d$ und $k \in \{1,\dots,(M + 1)^d\}$:  
\begin{align}
\label{fnet}
f_{\net,\bj,\mathbf{i}}(x) & = f_1^{(0)}(x). \nonumber
\intertext{wobei} \nonumber
f_k^{(l)}(x) & = f_{\mult}\Big(f_{2k - 1}^{(l + 1)}(x),f_{2k}^{(l + 1)}(x)\Big) \nonumber
\intertext{für $k \in \{1, 2, \dots, 2^l \}$ und $l \in\{0,\dots,s - 1\}, $ und}
f_k^{(s)}(x) & = f_{\id}(f_{\id}(x^{(l)} - x_{\mathbf{i}_k}^{(l)}))
\intertext{für $j_1 + j_2 + \dots + j_{l-1} + 1 \leq k \leq j_1 + j_2 + \dots + j_l$ und $1 \leq l \leq d$ und}
f_{|\bj|_1+ k}^{(s)}(x) & = f_{\mathrm{hat}, x_{\mathbf{i}_k}^{(k)}}(x^{(k)}) \nonumber
\intertext{für $1 \leq k \leq d$ und}
f_k^{(s)}(x) & = 1 \nonumber
\end{align} 
für $|\bj|_1 + d + 1 \leq k \leq 2^s.$
 
Da das neuronale Netz $f_{\net,\bj,\bi}$ aus mehreren neuronalen Netzen zusammengebaut wurde, lässt sich dadurch auch die Anzahl an Schichten und Neuronen pro Schicht durch diese Struktur erklären. Aus der rekursiven Definition~(\ref{fnet}) entnimmt man, dass $f_{\net,\bj,\bi}$ $s + 2$ verborgene Schichten, durch $s$-maliges Anwenden von $f_{\mult}$ und einer Anwendung von $f_{\mathrm{hat}}$ bzw.\@ $f_{\id}(f_{\id})$, hat. Da $f_{\mathrm{hat}}$ zwei verborgene Schichten besitzt, ergibt sich daraus die Anzahl an verborgenen Schichten von $f_{\net,\bj,\bi}$.
%Für die Anzahl an Neuronen für die jeweiligen Schichten können wir nur eine oberen Schranke angeben, da ... (TBD)
Die Anzahl der Neuronen pro verborgener Schicht von $f_{\net,\bj,\bi}$ ergeben sich wie folgt:
\begin{itemize}
\item Die erste verborgene Schicht enthält maximal $3 \cdot 2 \cdot 2^s = 6 \cdot 2^s$ Neuronen, da dies die erste verborgene Schicht von $f_{\mathrm{hat}}$ ist und maximal $(2^s)$-mal aufgerufen wird. 
\item Die zweite verborgene Schicht maximal $3 \cdot 4 \cdot 2^s = 12\cdot 2^s$ Neuronen, da dies die zweite verborgene Schicht von $f_{\mathrm{hat}}$ ist und maximal $2^s$-mal aufgerufen wird.
\item Die verborgenen Schichten $3,\dots,s + 2$ enthalten maximal   $2^{s+ 1}, 2^s, \dots, 2^3, 2^2$ Neuronen, da wir $s$-mal $f_{\mult}$ ineinander geschachtelt aufrufen. 
\end{itemize}  
Da man, wie in Kapitel~\ref{chap:1} bereits erwähnt, bei fully connected neuronalen Netzen die Gewichte der Verbindungen zwischen zwei Neuronen auf Null setzen kann, können auch nicht fully connected neuronale Netze in dieser Klasse enthalten sein. Daher liegt auch $f_{\net,\bj,\bi}$ nach Definition~\ref{def:nn} in $\mathfrak{N}(s + 2,\{24 \cdot (N + d)\}^{s + 2},\sigma)$, da die größte Anzahl an Neuronen in einer Schicht $$12 \cdot 2^s = 12 \cdot 2^{\lceil\log_2(N + d)\rceil} \leq 12 \cdot 2^{\log_2(N + d) + 1} = 24 \cdot (N + d)$$ ist. Weiterhin erkennt man durch die Zusammensetzung der neuronalen Netze, dass alle Gewichte im Betrag durch $c \cdot \max\{\frac{M}{2a}, R^2\}$ beschränkt sind, wobei $c > 0$ ist. 

\section{Bestimmung der Gewichte der Ausgabeschicht}
\label{subsec:2.2}

Wir definieren unseren Neuronale-Netze-Regressionsschätzer $\tilde{m}_n(x)$  durch:
\begin{equation}
\label{estimate}
\tilde{m}_n(x) = \sum_{\bi \in [M]^d} \sum_{\substack{\bj \in [N]^d\\|\bj|_1 \leq N}} a_{\mathbf{i},\bj} \cdot f_{\net,\bj,\mathbf{i}}(x),
\end{equation}
wobei $n$ die Größe unserer gegebenen Datenmenge~(\ref{dataset}) ist und wir die Koeffizienten $a_{\mathbf{i},\bj}$ durch Minimierung des Funktionals 
\begin{equation}
\label{min} 
\begin{split}
\varphi(a_{\bi,\bj}) & \coloneqq \frac{1}{n} \sum_{i = 1}^n|Y_i - \tilde{m}_n(X_i)|^2 + \frac{c}{n} \cdot \|(a_{\bi,\bj})\|_2^2
\end{split}
\end{equation}
für Regularitätsterm $\frac{c}{n}\|(a_{\bi,\bj})\|_2^2$ mit einer von $n$ unabhängigen Konstante $c > 0$ erhalten. 

Wir wollen Gleichung~(\ref{min}) als Gleichungssystem darstellen. Dafür definieren wir uns die Menge 
\begin{align*} 
\{U_s : s = 1,\dots,S\} = \Bigl\{f_{\net,\bj,\bi}(x) : \bi \in [M]^d \text{ und } |\bj|_1 \leq N \, \text{ mit }\, \bj \in [N]^d \Bigr\}
\end{align*}
wobei
$$ S \coloneqq \big|[M]^d\big| \cdot  \binom{N + d}{d} = (M + 1)^d \cdot \binom{N + d}{d}$$ die Kardinalität der Menge ist.
Diese Kardinalität erhalten wir mit einem Kombinatorik Argument.
Wir wissen, dass es insgesamt $(M + 1)^d$ Möglichkeiten gibt $d$ viele Zahlen aus einer Menge mit der Größe $(M + 1)$ zu ziehen mit Zurücklegen und da wir Vektoren betrachten und die Komponenten nicht vertauschbar sind, ist auch die Reihenfolge der Ziehung zu beachten.
Für jedes dieser $(M + 1)^d$ Möglichkeiten ist noch zu beachten, dass wir zusätzlich $d$-mal aus einer Menge mit $(N + 1)$ vielen Zahlen ziehen und gleichzeitig die Bedingung beachten müssen, dass die Summe der gezogen $d$ Elemente zwischen Null und $N$ liegt.
Gesucht ist also 
$$\bigg|\Bigl\{\bj \in [N]^d : |\bj|_1 \leq N \Bigr\}\bigg| \eqqcolon H.$$ 
Wir stellen fest, dass
\begin{equation*}
\begin{split}
& \Bigl\{\bj \in [N]^d : |\bj|_1 \leq N \Bigr\} \\
& = \Bigl\{\bj \in [N]^d : |\bj|_1 = 0 \Bigr\}
 \cup \Bigl\{\bj \in [N]^d : |\bj|_1 = 1 \Bigr\}
 \cup \dots 
 \cup \Bigl\{\bj \in [N]^d : |\bj|_1 = N\Bigr\}
\end{split}
\end{equation*}
gilt. Mit Lemma~\ref{lem:kombi} wissen wir, dass für $d, N \in \N$ und $k \in \N_0$ die Identität
$$\bigg|\Bigl\{\bj \in [N]^d : |\bj|_1 = k \Bigr\}\bigg| = \binom{d + k - 1}{k}$$ gilt.
Damit erhalten wir
$$|H| = \sum_{k = 0}^N \binom{N - 1 + k}{k} = \binom{N + d}{d},$$
mit der \emph{Hockey-Stick Identität}
$$\binom{n}{k} = \sum_{i = 0}^k \binom{n - k - 1 + i}{i} \quad (k, n \in \N  \text{ mit  } k < n).$$
Wir setzen nun 
$$ \mathbf{U} = (U_s(X_i))_{1\leq i \leq n,1\leq s \leq S} \quad \text{und} \quad \mathbf{Y} = (Y_i)_{i = 1,\dots,n}.$$
Im folgenden Lemma bestimmen wir den Koeffizientenvektor unseres Schätzers $\tilde{m}_n$.
\begin{lem}
Der Koeffizientenvektor unseres Schätzers $\tilde{m}_n$ aus Gleichung~(\ref{estimate}) ist die eindeutige Lösung des linearen Gleichungssystems 
\begin{equation}	
\label{les}
\bigg(\frac{1}{n}\mathbf{U}^T\mathbf{U} + \frac{c}{n} \cdot \mathbf{1} \bigg) \mathbf{a} = \frac{1}{n} \mathbf{U}^T\mathbf{Y}
\end{equation}
Hierbei ist $\mathbf{1}$ eine $S \times S$-Einheitsmatrix und $\ba \in \R^S$, wobei $\ba$ wie in Gleichung~(\ref{min}) den Ausdruck:
\begin{equation}
\label{eq:min}
\begin{split}
& \frac{1}{n} \sum_{i = 1}^n|Y_i - \tilde{m}_n(X_i)|^2 + \frac{c}{n} \cdot \sum_{\bi \in [M]^d} \sum_{\substack{\bj \in [N]^d\\|\bj|_1 \leq N}} a_{\mathbf{i},\bj}^2 \\
& = \frac{1}{n}(\mathbf{Y} - \mathbf{U}\mathbf{a})^T(\mathbf{Y} - \mathbf{U}\mathbf{a}) + \frac{c}{n} \cdot \mathbf{a}^T\mathbf{a}
\end{split}
\end{equation}
minimiert.
\end{lem}
\begin{proof}
Der Schätzer aus Gleichung~(\ref{estimate}) lässt sich umschreiben zu 
\begin{equation}
\label{umschreiben}
\tilde{m}_n(x) = \sum_{s = 1}^S a_s \cdot U_s(x)
\end{equation}
mit $(a_s)_{s = 1,\dots,S} = \mathbf{a}\in \R^S$. 

Es gilt $\mathbf{a}^T\mathbf{U}^T\mathbf{Y} = \mathbf{Y}^T\mathbf{U}\mathbf{a}$, da dieser Ausdruck eine reelle Zahl und damit insbesondere symmetrisch ist. Damit erhalten wir für die rechte Seite aus Gleichung~(\ref{eq:min}):
\begin{equation}
\label{eq:matrix}
\begin{split}
& \frac{1}{n}(\mathbf{Y} - \mathbf{U}\mathbf{a})^T(\mathbf{Y} - \mathbf{U}\mathbf{a}) + \frac{c}{n} \cdot \mathbf{a}^T\mathbf{a} \\
& = \frac{1}{n}(\mathbf{Y}^T\mathbf{Y} - \mathbf{Y}^T\mathbf{U}\mathbf{a} - \mathbf{a}^T\mathbf{U}^T\mathbf{Y} + \mathbf{a}^T\mathbf{U}^T\mathbf{U}\mathbf{a}) + \frac{c}{n} \cdot \mathbf{a}^T\mathbf{a} \\
& = \frac{1}{n}(\mathbf{Y}^T\mathbf{Y} - 2\mathbf{Y}^T\mathbf{U}\mathbf{a}) + \mathbf{a}^T\bigg(\frac{1}{n} \mathbf{U}^T\mathbf{U} + \frac{c}{n} \cdot \mathbf{1}\bigg) \mathbf{a}.
\end{split}
\end{equation} 
Die Matrix $\mathbf{U}^T\mathbf{U} \in \R^{S \times S}$ ist positiv definit, denn aufgrund der Rechenregeln der Transponierten und des Standardskalarprodukts sowie der positiven Definitheit des Standardskalarprodukts gilt für alle $x \in \R\setminus\{0\}$:
$$\langle x, \mathbf{U^T}\mathbf{U} x\rangle = \langle \mathbf{U} x, \mathbf{U} x\rangle \geq 0.$$
Zudem wissen wir dass $\frac{c}{n}\mathbf{1}$ durch die Wahl von $c$ nur positive Eigenwerte besitzt und damit positiv definit ist.  
Daher wissen wir, dass die Matrix
$$\mathbf{A} \coloneqq \frac{1}{n}\mathbf{U}^T\mathbf{U} + \frac{c}{n} \cdot \mathbf{1}$$ positiv definit und insbesondere invertierbar ist, da die Eigenwerte positiv sind. Das folgt daraus, dass die ohnehin schon positiven Eigenwerte von $\frac{1}{n}\mathbf{U}^T\mathbf{U}$ um $\frac{c}{n}$ verschoben werden und damit positiv bleiben. Zudem ist die Matrix $\mathbf{A}$ symmetrisch. 
Mit $$\mathbf{b} = \frac{1}{n} \cdot \mathbf{A}^{-1}\mathbf{U}^T\mathbf{Y} \in \R^S$$ und $$\mathbf{b}^T\mathbf{A}\mathbf{a} = \mathbf{a}^T\mathbf{A}\mathbf{b} = \frac{1}{n} \cdot \mathbf{a}^T\mathbf{U}^T\mathbf{Y} \in \R,$$ was aus der Symmetrie von $\mathbf{A}$ folgt, erhalten wir mit 
$$ \mathbf{Y}^T\mathbf{U}\mathbf{a} = \frac{n}{2} \cdot \bb^T\bA\ba + \frac{n}{2} \cdot \ba^T\bA\bb $$
mithilfe von $$0 = \bb^T\bU^T\bY - \bY^T\bU\bb$$
die Gleichung 
\begin{equation}
\begin{split}
 \mathbf{Y}^T\mathbf{U}\mathbf{a} & = \frac{n}{2} \Big(\bb^T\bA\ba + \ba^T\bA\bb\Big) = \frac{n}{2}\Big( \bb^T\bA\ba + \ba^T\bA\bb - \frac{1}{n}\bb^T\bU^T\bY + \frac{1}{n}\bY^T\bU\bb \Big)\\
 & = \frac{n}{2} \Big( \bb^T\bA\ba + \ba^T\bA\bb - \frac{1}{n}\bb^T\bA\bA^{-1}\bU^T\bY + \frac{1}{n}\bY^T\bU\bb \Big)\\
& = \frac{n}{2} \Big(\bb^T\bA\ba + \ba^T\bA\bb - \bb^T\bA\bb + \frac{1}{n^2}\bY^T\bU\bA^{-1}\bU^T\bY \Big).
\end{split}
\end{equation}

Damit erhalten wir in Gleichung~(\ref{eq:matrix}):
\begin{equation*}
\begin{split}
& \frac{1}{n}(\mathbf{Y}^T\mathbf{Y} - 2\mathbf{Y}^T\mathbf{U}\mathbf{a}) + \mathbf{a}^T\bigg(\frac{1}{n} \mathbf{U}^T\mathbf{U} + \frac{c}{n} \cdot \mathbf{1}\bigg) \mathbf{a} \\
& = \frac{1}{n}(\mathbf{Y}^T\mathbf{Y} - 2\mathbf{Y}^T\mathbf{U}\mathbf{a}) + \mathbf{a}^T \bA \mathbf{a} \\
& = \frac{1}{n}\bY^T\bY - \bb^T\bA\ba - \ba^T\bA\bb + \bb^T\bA\bb - \frac{1}{n^2}\bY^T\bU\bA^{-1}\bU^T\bY + \ba^T\bA\ba \\
& = (\mathbf{a} - \frac{1}{n} \cdot \mathbf{A}^{-1}\mathbf{U}^T\mathbf{Y})^T \mathbf{A} (\mathbf{a} - \frac{1}{n} \cdot \mathbf{A}^{-1} \mathbf{U}^T\mathbf{Y}) + \frac{1}{n}\mathbf{Y}^T\mathbf{Y} - \frac{1}{n^2}\mathbf{Y}^T\mathbf{U}\mathbf{A}^{-1}\mathbf{U}^T\mathbf{Y}.
\end{split} 
\end{equation*} 
Die letzte Gleichung wird für $\mathbf{a} = \frac{1}{n} \cdot \mathbf{A}^{-1}\mathbf{U}^T\mathbf{Y}$ minimal, 
da wir wissen, dass $\mathbf{A}$ positiv definit ist und damit $x^T\mathbf{A}x > 0$ für alle $x \in \R^S$ mit $x \neq 0$ gilt und $(\mathbf{a} - \mathbf{b})^T\mathbf{A}(\mathbf{a} - \mathbf{b}) = 0$ genau dann, wenn $\mathbf{a} = \mathbf{b}$ gilt.
Dies zeigt also, dass der Koeffizientenvektor unseres Schätzers~(\ref{estimate}) die eindeutige Lösung des linearen Gleichungssystems~(\ref{les}) ist.
\end{proof}
\begin{bemnumber}
\label{mtildebeschraenkt}
Da der Koeffizientenvektor $\mathbf{a}$ die Gleichung~(\ref{eq:min}) minimiert, erhalten wir, wenn wir den Koeffizientenvektor gleich Null setzen:
$$\frac{1}{n}(\mathbf{Y} - \mathbf{U}\mathbf{a})^T(\mathbf{Y} - \mathbf{U}\mathbf{a}) + \frac{c}{n} \cdot \mathbf{a}^T\mathbf{a} \leq \frac{1}{n} \sum_{i = 1}^n Y_i^2,$$
was uns erlaubt eine obere Schranke für den absoluten Wert unserer Koeffizienten abzuleiten. Daraus können wir folgern, dass unser Neuronale-Netze-Regressionsschätzer $\tilde{m}_n$ beschränkt ist, da die neuronalen Netze $f_{\net,\bj,\bi}$ nach Konstruktion ebenfalls beschränkt sind.
\end{bemnumber}